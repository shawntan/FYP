\tikzstyle{background}=[rectangle,
	fill=gray!10,
	inner sep=0.2cm,
	rounded corners=5mm]


\tikzstyle{post}=[circle,
	thick,
	minimum size=0.75cm,
	draw=blue!80,
	fill=blue!20]
% The measurement vector is represented by an orange circle.
\tikzstyle{visit}=[circle,
	thick,
	minimum size=0.75cm,
	draw=orange!80,
	fill=orange!25]

\begin{tikzpicture}[>=latex,text height=1.5ex,text depth=0.25ex]
    % "text height" and "text depth" are required to vertically
    % align the labels with and without indices.
  
  % The various elements are conveniently placed using a matrix:
  \matrix[column sep=0.3cm] {
    % First line: Control input
    	&
		\node (e0)				{$\cdots$};&
		\node (e1)	[post]			{$\rho_1$}; &
		\node (e2)	[visit]			{$\rho_2$}; &
		&&
		\node (e3)	[post]			{$\rho_3$}; &
		\node (e4)	[post]			{$\rho_4$}; &
		&&
		\node (e5)	[visit]			{$\rho_5$}; &
		\node (e6)	[post]			{$\rho_6$}; &
		\node (e7)	[visit]			{$\rho_7$}; &
		\node (e)				{$\cdots$};&
		&
        \\
	};
    
    % The diagram elements are now connected through arrows:

	\path[-]
		(e0) edge[thick]	(e1)
		\foreach \e in {1,2,3,4,5,6}{
			let \n1={int(\e+1)} in (e\e) edge[thick] (e\n1)
		}
		(e7) edge[thick]	(e)
	;
	\begin{pgfonlayer}{background}
		\only<1>{\node [background,fit=(e1) (e3)] {};}
		\only<2>{\node [background,fit=(e3) (e4)] {};}
		\only<3>{\node [background,fit=(e4) (e6)] {};}
    \end{pgfonlayer}


\end{tikzpicture}
