With the advent of Web 2.0, sites with forums, or similar thread-based
discussion features are increasingly common.  Our goal in this thesis
is to create an algorithm that can predict when updates in such
threads will occur.
%Just recently, Google+ has just
\begin{table}\label{table:web20}
	\makebox[\textwidth][c]{
	{\footnotesize
	\begin{tabular}{|l|c|c|c|c|c|c|c|c|c|c|}
		\hline
			\input{tables/web20}
		\hline
	\end{tabular}
	~\\
	}
	}
	{\footnotesize
\caption{Features of popular Web 2.0 sites}
	\begin{tabular}{l l}
		T &= Twitter mentions\\
	 FB L &= Facebook Likes \\
		FB S &= Facebook Shares\\
	G +1 &= Google +1\\
		   L&= Likes (Local) \\
   		DL &= Dislikes (Local) \\
			C &= Comments \\
		PV &= Page Views \\
   Follows &= Site-local feature for keeping track of user's activities
	\end{tabular}
}
\end{table}

Table \ref{table:web20} shows us that many of the popular Web 2.0
sites have comment features. This suggests that content on the web is
increasingly being created by users alongside content providers.

%we are predicting web 2.0 updates
In an increasing number of cases, news travels more quickly through
online community discussions than through traditional media. Users also
typically discuss purchased products bought online on forums,
and companies that want to get timely feedback about their product
should turn to data mined from such sites.

 A naive way of getting timely updates is to repeatedly poll the websites 
to check for the presence of new posts. However, the large number of 
pages in popular forum sites make this task infeasible in practice, 
% Min: need to talk about DOS. Most sites don't mind occasional
% bandwidth use but too much is not a good thing.  See white-hat rules
% about spiders.
A simple method to reduce the amount of polling needed is to use the
previous time differences between previous posts to estimate the
arrival of the next one, and to abstain from polling until the
estimated time.
% Min: you need to discuss how well this simple baseline does and whether it is actually adopted.

% Min: this needs a new paragraph; it's about a different topic now.
A key observation in our work is that the contents of the thread may
also influence the discussion and hence the rate of commenting.  We
believe that the content of the thread has information that can give a
better estimate of the time interval between the last post and a new
one.

% Min: tie the ``hypothetical'' example with actual posts.  Also, your
% introduction needs to actually carry through to the final
% implementation and experimentation where you use such features to
% estimate properly.
 For example, a thread in a technical forum about a Linux distribution may start 
out as a question. Subsequent questions that attempt to either clarify or expand 
on the original question may then be posted, resulting in a quick flurry of 
messages. Eventually, a more technically savvy user of the forum may come up 
with a solution, and the thread may eventually slow down after a series of 
messages thanking the problem solver. 

% Min: your introduction needs some work.
% Towards the end you need.  Please restructure the below.
% 1) a clear problem statement
% 2) statements of the contribution of your work
% 3) A navigation paragraph that describes how the rest of the report
% is going to be structured, especially if it deviates from ``normal''
% structure.

Let us define all such thread-based discussion styled sites as forums. Ideally, 
an incremental crawler of such user-generated content should be able to maintain 
a fresh and complete database of content of the forum that it is monitoring.  
However, doing so with the previously mentioned naive method would (1) incur 
excessive costs when downloading un-updated pages, and (2) raise the possibility 
of the web master blocking the requester's IP address.

