With the advent of Web 2.0, sites with forums, or similar thread-based
discussion features are increasingly common.  
% Jesse: You just made a claim but its not immediately backed-up with any
% supporting statements. Mention of Table 1.1 should probably go here.
Our goal in this thesis
is to create an algorithm that can predict when updates in such
% Jesse: At this juncture, its still unclear what you mean by "updates in such
% threads".
threads will occur.
\begin{table}
	\makebox[\textwidth][c]{
	{\footnotesize
	\begin{tabular}{|l|c|c|c|c|c|c|c|c|c|c|}
		\hline
			\input{tables/web20}
		\hline
	\end{tabular}
	~\\
	}
	}
	{\footnotesize
\caption{Features of popular Web 2.0 sites}
\label{table:web20}
	\begin{tabular}{l l}
		T &= Twitter mentions\\
	 FB L &= Facebook Likes \\
		FB S &= Facebook Shares\\
	G +1 &= Google +1\\
		   L&= Likes (Local) \\
   		DL &= Dislikes (Local) \\
			C &= Comments \\
		PV &= Page Views \\
   Follows &= Site-local feature for keeping track of user's activities
	\end{tabular}
}
\end{table}

Table \ref{table:web20} shows us that many of the popular Web 2.0
sites have comment features. This suggests that content on the web is
increasingly being created by users alongside content providers.

%we are predicting web 2.0 updates
In an increasing number of cases, news travels more quickly through
online community discussions than through traditional media. Users also
typically discuss purchased products bought online on forums,
and companies that want to get timely feedback about their product
should turn to data mined from such sites.

Web crawling is largely IO-bound, a large portion of the time spent crawling is 
% Jesse: awkard phrasing: "a large portion of X is waiting for the server".
% You may want to rephrase "is" to something else.
waiting for the server to supply a response to the request issued by the 
crawler. However, for sites with a large number of pages (like popular forum 
sites), make this infeasible in practice. 
% Jesse: Makes what infeasible? What is the noun you are referring to?
On top of the usual requests it has, 
% Jesse: Another problem with anaphor resolution of "it". You used "sites" not
% "site" earlier.
it then has to deal with repeated requests from such a crawler. Most sites do 
% Jesse: Another anaphor resolution problem for "it". Please check this throughout
% the thesis.
not mind some additional bandwidth, but if it gets excessive, it may be 
% Jesse: here "it" refers to crawling. When was that last mentioned?
construed as a Denial-of-Service attack. At best, the site may deny any further 
% Jesse: "site" or "sites". Use one or the other. Do not mix.
requests from the crawler, and at worst the large number of requests may bring 
% Jesse: I'm not sure if one crawler can bring down an entire site. It takes a
% significant number of requests per second to bring down a site, one that you
% say "is popular", so we can assume that they will have alotted a non-trivial
% amount of bandwidth in the first place. I'm just saying that this assertion
% sounds unrealistic. The crawler doesnt need to bring down the site for this
% situation to be a problem for the site and/or crawler. 
down the site.
% bandwidth use but too much is not a good thing.  See white-hat rules
% about spiders.

A simple method to reduce the amount of polling needed is to use the
% Jesse: "amount of polling needed". I dont believe you have established what
% you mean by "needed". 
average time differences between previous posts to estimate the
% Jesse: "between previous posts" is ambiguous. Between which ones of the
% previous posts?
arrival of the next one, and to abstain from polling until the
estimated time.
% Min: you need to discuss how well this simple baseline does and whether it is 
% actually adopted.
% Jesse: agreed. please remember to include the results for this baseline in
% the experiment section.

A key observation in our work is that the contents of the thread may
also influence the discussion and hence the rate of commenting.  
% Jesse: Can you give a sneak peak (like one enlightening example) for this
% observation?
We
believe that the content of the thread has information that can give a
better estimate of the time interval between the last post and a new
one.

% Min: tie the ``hypothetical'' example with actual posts.  Also, your
% introduction needs to actually carry through to the final
% implementation and experimentation where you use such features to
% estimate properly.
% Jesse: The paragraph below comes too late. The doubt in the reader's mind
% appears immediately after reading "A key observation...", but you only
% mention this example one paragraph later. If you follow Min's suggestion and
% use actual posts, you can refer make label those posts with figure numbers
% and refer to the posts in the text.
For example, a thread in a technical forum about a Linux distribution may start 
out as a question. Subsequent questions that attempt to either clarify or expand 
on the original question may then be posted, resulting in a quick flurry of 
messages. Eventually, a more technically savvy user of the forum may come up 
with a solution, and the thread may eventually slow down after a series of 
messages thanking the problem solver. 

% Min: your introduction needs some work.
% Towards the end you need.  Please restructure the below.
% 1) a clear problem statement
% 2) statements of the contribution of your work
% 3) A navigation paragraph that describes how the rest of the report
% is going to be structured, especially if it deviates from ``normal''
% structure.
Let us define all such thread-based discussion styled sites as forums. Ideally, 
an incremental crawler of such user-generated content should be able to maintain 
a fresh and complete database of content of the forum that it is monitoring.  
However, doing so with the previously mentioned naive method would (1) incur 
excessive costs when downloading un-updated pages, and (2) raise the possibility 
of the web master blocking the requester's IP address.
% Jesse: You need to make a better transition to the next paragraph. How does
% this situation lead you to have the following goal? Why is this goal the
% logical step (or one of the logical steps) to solve the situation. Note that
% Im not asking you to write the answers to these questions as text in the
% report. These questions are meant to make you think about what to actually
% write. Same goes with all of my other questions.

Our high level goal: To come up with a suitable algorithm for revisiting user 
% Jesse: "To come up" is not good English
% Jesse: "for revisiting"? Please use a consistent terminology
discussion threads, based on the discussion content in the thread. In this 
project, we focus on forum threads. We demonstrate three different methods for 
achieving this using regression methods, and also propose a new metric for 
measuring the timeliness of such a model that balances between the model's 
timeliness and bandwidth consumption.
% Jesse: You are burying the lead. It seems to me that you have major
% contributions in this thesis, but you are hiding them inside this tiny
% little paragraph. Make them stand-out. Use a numbered list. Make it clear
% that these are your contributions, especially if they are novel. And if they
% are, you should mention that they're novel whenever appropriate.

In Chapter 2, we explore the related work dealing with predicting web page 
updates and metrics to measure the performance of such algorithms. 
% Jesse: awkward super long noun above "dealing with [noun phrase]"
In Chapter 3, 
we discuss the methods that we have come up with to tackle the problem, while 
% Jesse: "we have come up with" rephrase
Chapter 4 describes the metrics we propose for measuring the performance of 
revisitation algorithms. 
% Jesse: "Revisitation algorithms"? 
In Chapter 5, we perform experiments on a dataset 
extracted from \url{avsforum.com}, 
% Jesse: you could mention a little more about the significance of the results
% and the size of the data set here.
and show that our models perform better than 
an average revisitation baseline. Chapter 6 then discusses our contributions, 
and possible avenues of future work.
