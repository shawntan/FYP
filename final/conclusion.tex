With the increasing number of sites leveraging user-generated content, a method 
for predicting the updates of such sites needs to be created in order for an 
incremental web crawler to effectively crawl the site. Our high level goal: to 
predict the posting behaviour of users to such sites.

While this primary goal has many challenges, in this report, we have chosen to 
address challenges specific to forum threads. We want to predict, given content 
of the current thread, the time at which a user would post to the thread. We 
evaluate three different machine learning approaches: Two offline algorithms, 
one that only takes into account only the latest window, and another that 
accounts for past windows, with decreasing weightage. And an online algorithm, 
that uses gradient descent to update its weights every time a new post is 
observed.

Overall, our evaluation shows that our methods work better than the baseline, 
which was to revisit the thread at the average time interval. These are 
promising results, and more can be done to improve upon them. 

There are, however, limitations with the current methods, such as....


\section{Contributions}
We have made the following contributions with our work:
\begin{enumerate}
\item Provide comparable evaluation metrics that can be parameterised, depending 
on the evaluators' priority: freshness or bandwidth.

\item Three different prediction methods using machine learning.
\item The effectiveness of using content and time differences for post 
prediction.
\end{enumerate}

\section{Future Work}

With the proposed methods still having many shortcomings when predicting new 
posts, or providing insight into what causes post arrival times, it leaves much 
room for future work to be done.

\subsection{Using Natural Language Processing (NLP) techniques}
\cite{Wang}

\subsection{Topic modelling}
\cite{Gonzalez2005,Hsu2006}

\subsection{Leveraging context}


\subsection{Using online learning techniques}



