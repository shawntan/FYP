\section{Introduction}
This report details what has been completed since the proposal of the project.
The hope is that by modeling the topics, we will be able to infer something 
about the time which the post will arrive.

We shall call the time differences between posts $\dt$. For the rest of this 
report, $\W$ represents a document, in particular, the collection of words that 
the document contains, and $\Z$ represents a topic that a document belongs to.

In the next section, we discuss the details of the project that have been 
completed and some of the complications. The last section will detail the 
remaining tasks left to be done.
\section{Completed Tasks}
\subsection{Data: avsforum.com}
%Window
Our dataset was obtained from from \url{avsforum.com} and stored in a 
tab-delimited format, with each line representing a new post. For each post, we 
extracted the following:
\begin{description}
	\item[Timestamp] Time in seconds-from-epoch format.
	\item[Author] Username of the author of the post.
	\item[Main content] The main textual content of the post.
\end{description}
%TODO REwrite?
For the our purposes, we pre-process the data such that each instance is 
comprised of $k$ posts. We have found that Support Vector Regression gives the 
best results when trying to infer the time of the next post arrival when $k=15$.


\subsection{Latent Dirichlet Allocation}
We used Gibbs sampling to perform LDA for 2 to 10 topics. For a small subset of 
threads in the forum. Treating each window as a document, we attempted to find 
topics that the words in the documents belonged to. The following are some 
interesting results.

Since it is a forum that deals largely with audio visual equipment, a large 
portion of the posts tend to comprise of people seeking to troubleshoot faulty 
equipment.  One of the topics LDA found had the following words:
\begin{verbatim}
player	dvd	get	one	would	use	play	audio	like	samsung	blu	hdmi	
sound	work	problem	ray	also	tri	think	firmwar	set	disc	know	issu	
new	see	updat	look	want	soni	good	better	time	connect	even	
thankstill	need	say	buy	receiv	back	price	output	oppo	make	seem	
video	unit	realli	support	well	soundbar	differcould	vizio	optic	
format	movi	analog	come	cabl	via	got	anyon	much	sinc	denon	
may	sure	display	decod	thing	doesnbest	way	bar	fix	pcm	post	first	
watch	channel	hope	test	digit	said	far	take	panason	right	
anoth	bitstream	return	abl	file
\end{verbatim}

These seem to be the keywords that are consistent with such posts -- when users 
describe `issues' and `problems' with their `firmware', and decide if they 
should `return' the goods.

Another distinct topic that was revealed had to do with positive sentiment, and 
contain words such as `thanks' or `nice'.
\begin{verbatim}
speaker	like	amp	sub	sound	one	####	use	###	would	get	listen	room	
music	good	also	look	power	two	system	realli	set	think	much	
know	post	channel	better	new	pair	audio	great	thank	bass	time	
hear	heard	differ	even	make	see	high	want	say	well	center	
come	level	surround	back	could	still	price	current	wire	need	
home	jamo	dsp	first	play	tri	never	sure	nice	love	end	
output	peopl	design	receiv	thread	focu	run	cabl	front	review	
theater	test	#.#	may	anyon	year	thx	got	hsu	watt	legaci	compar	
subwoof	setup	main	bit	work	right	klipsch	somethdual	impress	guy
\end{verbatim}


However, what we want to study is if these differences in content reflect a 
difference in the arrival times of the next post.

\subsection{Distribution of $\dt$}
Since we have a way of obtaining $\P{\Z}{\W}$, we need $\P{\dt}{\Z}$ in order to 
have a way to predict $\dt$. To have a sense of what this looks like, we have 
binned the $\dt$ into bins of 20 minutes, and plotted their frequency based on 
the topics we extracted. A document is said to be coming from topic $z$ if 
$\max_{z'} \P{z'}{\W} = z$
The idea was to see if there was a distinguishable $\dt$ distribution given 
different topics. See Figure \ref{fig:time_dist}.
\begin{figure}
\centering
\begin{subfigure}[b]{0.5\textwidth}
	\centering
	\includegraphics[width=\textwidth]{/home/shawn/Desktop/predict-forum-pgm/graphs/hist_4_topics/001.png}
	\caption{Topic 1}
\end{subfigure}%
~ %add desired spacing between images, e. g. ~, \quad, \qquad etc. 
  %(or a blank line to force the subfigure onto a new line)
\begin{subfigure}[b]{0.5\textwidth}
	\centering
	\includegraphics[width=\textwidth]{/home/shawn/Desktop/predict-forum-pgm/graphs/hist_4_topics/002.png}
	\caption{Topic 2}
\end{subfigure} \\
\begin{subfigure}[b]{0.5\textwidth}
	\centering
	\includegraphics[width=\textwidth]{/home/shawn/Desktop/predict-forum-pgm/graphs/hist_4_topics/003.png}
	\caption{Topic 3}
\end{subfigure}%
~%
 %
\begin{subfigure}[b]{0.5\textwidth}
	\centering
	\includegraphics[width=\textwidth]{/home/shawn/Desktop/predict-forum-pgm/graphs/hist_4_topics/004.png}
	\caption{Topic 4}
\end{subfigure}
\caption{Time distributions for $k = 4$}\label{fig:time_dist}
\end{figure}

Unfortunately, the time distributions did not appear very distinct from each 
other. More tests need to be performed to see if using the expected value from 
this currently gathered data will reflect anything.
\section{Work In Progress}
Including the previous post topic, we can make a better estimate of the $\dt$ 
distribution of the current post.
\[
	p(\Z_t|\Z_{t-1},\W)=
	\frac{p(\Z_{t-1}|\Z_t) \cdot p(\Z_t|\W_t) \cdot p(\W_t)}{
	p(\Z_{t-1})\cdot p(\W)}
\]

This essentially makes it similar to a Hidden Markov Chain, since the document 
topic is dependent on the previous document's topic, and the observations 
produced are the text. This is, however, dependent on whether the distribution 
of $\dt$ is different for each topic.

As seen in Figure \ref{fig:time_dist}, which shows one example of LDA produced 
topics, the time distributions are not distinguishable. This may suggest that 
this approach may not be feasible, and more work needs to be done.

