
\section{Potential errors}
To be thorough, let us also enumerate the types of errors that a model making 
predictions could encounter.

The model can potentially make a prediction such that the next visit comes 
before the arrival of the next post. The predictions being made are the $\dt$ 
between the posts, rather than the visitation times, hence, it is possible for 
the model to make a prediction that occurs before the current time. An erroneous 
prediction can also cause the crawler to come in before the next post (two, or 
more, visits, but nothing new fetched). Errors of this type waste bandwidth, 
since the crawler will make an unnecessary visit to the page.

Another type of error would have the prediction causing the next visit to come 
some time after a post. Since most predictions are almost never fully accurate, 
there will be some time between the post is made and the page is fetched. These 
errors are still relatively acceptable, but the time difference between the post 
arriving and the visit should be minimised. The visit could also come more than 
one post later. Errors of this kind incur a penalty on the freshness of the 
data, more so than the after one post, especially if the multiple posts are far 
apart time-wise.

In the following experiments, the threads chosen from our extracted dataset are 
those with a 100 to 1000 posts. This amounted to 97 threads. The first 75\% of 
the thread was used as training data, while the remaining 25\% was used as test 
data. We used Support Vector machines for this regression task, employing a 
Radial Basis Function kernel as our learning algorithm. 

%include diagrams

\section{Evaluation metrics}

\begin{figure}
	\begin{center}
	
\tikzstyle{background}=[rectangle,
	fill=gray!10,
	inner sep=0.2cm,
	rounded corners=5mm]


\tikzstyle{post}=[circle,
	thick,
	minimum size=1.2cm,
	draw=blue!80,
	fill=blue!20]
% The measurement vector is represented by an orange circle.
\tikzstyle{visit}=[circle,
	thick,
	minimum size=1.2cm,
	draw=orange!80,
	fill=orange!25]

\begin{tikzpicture}[>=latex,text height=1.5ex,text depth=0.25ex]
    % "text height" and "text depth" are required to vertically
    % align the labels with and without indices.
  
  % The various elements are conveniently placed using a matrix:
  \matrix[column sep=0.4cm] {
    % First line: Control input
    	&
		\node (e0)				{$\cdots$};&
		\node (e1)	[post]			{$\rho_1$}; &
		\node (e2)	[visit]			{$\rho_2$}; &
		&&
		\node (e3)	[post]			{$\rho_5$}; &
		\node (e4)	[visit]			{$\rho_6$}; &
		&&
		\node (e)				{$\cdots$};&
		&
        \\
	};
    
    % The diagram elements are now connected through arrows:

	\path[-]
		(e0) edge[thick]	(e1)
		\foreach \e in {1,2,3}{
			let \n1={int(\e+1)} in (e\e) edge[thick] (e\n1)
		}
		(e4) edge[thick]	(e)
	;
	\begin{pgfonlayer}{background}
		\only<1>{\node [background,fit=(e1) (e3)] {};}
		\only<2>{\node [background,fit=(e2) (e4)] {};}
    \end{pgfonlayer}


\end{tikzpicture}

	\caption{An example of a series of events used in our evaluation.}
	\end{center}
\end{figure}
\begin{figure}
	\begin{center}
	\input{diagrams/t_score_diag}
	\caption{An example of a series of events used in our evaluation.}
	\end{center}
\end{figure}

We use \emph{Mean Absolute Percentage Error} (MAPE), to measure the performance 
of the learnt model. This value is given by
\[
	\frac{1}{N}\sum^N_{i=1}\left|\frac{A_i-F_i}{A_i}\right|
\]
where $A_i$ is the actual value, and $F_i$ is the forecasted value for the 
instance $i$. Realistically, the model would not be able to come into contact 
with every possible window, since chances are it will make an error that causes 
 %explain error in a new section (before this one) ?
it to visit a thread late, causing it to miss two posts or more. This value does 
not reflect how well the model will do in a real-time setting, but gives an idea 
of how far off the model is given a window. 

We also want to know the \emph{timeliness} of the model's visits. Yang et. al.  
\cite{Yang2009} has a metric for measuring this. Taking $\Delta t_i$ as the time 
difference between a post $i$ and it's download time, the timeliness of the 
algorithm is given by
\[T = \frac{1}{N} \sum^{N}_{i=1}\Delta t_i\]
A good algorithm would give a low $T$-score. However, a crawler that hits the 
site repeatedly performs well according to this metric. The authors account for 
this by setting a bandwidth (fixed number of pages per day) for each iteration 
of their testing. In our experimental results, we also take into account the 
number of page requests made in comparison to the number of posts. %ratio?

\begin{align*}
	\begin{array}{l@{\mskip\thickmuskip}l}
	Pr_{miss} &=  \dfrac{%
		\sum^{N-k}_{i=1} \left[\Theta_{ref\_hyp} (i,k)\right]%
	}{%
		\sum^{N-k}_{i=1} \left[\Delta_{ref} (i,k)\right]%
	}\\
	 & \\
	Pr_{fa} &= \dfrac{%
		\sum^{N-k}_{i=1} \left[\Psi_{ref\_hyp} (i,k)\right]%
	}{N-k}
	\end{array}
	\begin{array}{l@{\mskip\thickmuskip}l}
		\Delta_{ref}(i,k) &= \left\{ \begin{array}{l l}
				1, & \text{if }r(i,k) > 0 \\
				0, & \text{otherwise} 
		\end{array} \right.\\
		\Theta_{ref\_hyp}(i,k) &= \left\{ \begin{array}{l l}
				1, & \text{if ends with post} \\
				0, & \text{otherwise} 
		\end{array} \right.\\
		\Psi_{ref\_hyp}(i,k) &= \left\{ \begin{array}{l l}
				1, & \text{if ends with visit}\\
				0, & \text{otherwise} 
		\end{array} \right.\\
	\end{array}
\end{align*}
Split the $Pr_{error}$ measure into 3 parts:

Probability of having visits before the first post in the window.
Probability of having more visits than posts after the first post.
Probability of having more posts than visits after the first post.



Viewing the posts made during the thread's lifetime as segmentations of the 
thread, and the visits made as hypotheses of where the segmentations are, we use 
the $\prerror$ metric from Georgescul et. al., 2006 as a measure of how close 
the predictions are to the actual posts. An example can be seen in Figure 
\ref{prerror}.



Function that gives me:

Increase in visit to post ratio		increase
Increase in interval				increase
Increase between post and visit		increase
(0,$\infty$ or 1)

Increase between visit and visit	decrease
($\infty$ or 1,0)
Increase between post and post		increase
(0,$\infty$ or 1)

\[
	\begin{array}{l l}
	T = \dfrac{%
		\sum_{e=1}^{|E|-1} \Psi(e_t,e_{t+1}) %
	}{|E|-1} &
		\Psi(e_t,e_{t+1}) = \left\{\begin{array}{l l}
				1-e^{-(e_{t+1} - e_t)}	& \text{if post,visit}\\
				e^{-(e_{t+1} - e_t)}			& \text{if visit,visit}\\
		\end{array}\right.
\end{array}
\]

\begin{figure}
\[
	\uparrow~~p_1,\underbrace{~~p_2,~~p_3,~~~p_4,\uparrow}_{\text{sliding window}}\uparrow
\]
\caption{An example of the sliding window metric. The metric is made up of two components: First, the probability that, given at least one post is present in the window, there are more visits than post. Secondly, the probability that there are more posts than visits for a given window. The weighted sum of this gives the overall $\prerror$}\label{prerror}
\end{figure}


\section{Normalising the $T$-score and Visit/Post ratio}
We normalise the $T$-score to get a comparable metric across all the threads. In 
order to do this, we consider again the thread posts and visits as a sequence of 
events. We then define the \emph{lifetime}, denoted as $l$, of the thread as the 
time between the first post and the last post. Any visits that occur after the 
last post are ignored.

We then consider the worst case in terms of timeliness, or misses. This would be 
the case where the visit comes at the end, at the same time as the post. So we 
get a value $T_{\max}$ and $P_{\text{miss}}$ such that
\[
	P_{\text{miss}} = \frac{T}{T_{\max}} = \dfrac{N \cdot T}{\sum_p l - p_t}
\]

It is difficult to consider the worst case in terms of false alarms, or visits 
that retrieve nothing. There could be an infinite number of visits made if we 
are to take the extreme case. In order to get around this, we consider discrete 
time frames in which a visit can occur. Since for this dataset, our time 
granularity is in terms of minutes, we shall use minutes as our discrete time 
frame.

With this simplified version of our series of events, we can then imagine a 
worst-case performing revisit policy that visits at every single time frame.  
This gives us
\[
	P_{\text{FA}} = \frac{|V|}{l - |P|}
\]
where $l$ is in units of our specified discrete time frame.


