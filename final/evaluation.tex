In our project, the dataset we used was crawled from 
\url{http://www.avsforum.com/f/}. The forum dealt mainly with Audio-Visual 
equipment, with discussions mainly about technical details, offers and people 
showing off their DIY projects.

The forum was chosen from the list which \outcite{Yang2009} provided in their 
paper. The forum users use mainly proper English, which made removing stopwords 
and stemming simpler.

We crawled 4,158 threads, with a total of 1,002,225 posts. A distribution of how 
the length of threads are distributed can be seen in Figure \ref{fig:len_dist}.  
Threads with 1 to 10 posts already make up half the number of collected threads.
The distribution of the time differences are shown in Figure 
\ref{fig:time_dist}. In both the figures, the right-hand-side cutoff was set at 
1,000 due to the negligible number of items to the right of the cutoff.


\putgraphic{diagrams/len_dist.png}{Distribution of thread 
length}{len_dist}{0.75}
\putgraphic{diagrams/time_dist.png}{Distribution of $\dt$}{time_dist}{0.75}

\section{Experiment setup}
The first 75\% of the thread was used as training data, while the remaining 25\% 
was used as test data. We used Support Vector Regression for this regression 
task, employing a Radial Basis Function kernel as our learning algorithm. 

This setup is used in both the tuning of parameters and for the full evaluation 
of the entire dataset.
\begin{figure}
	\input{diagrams/exp_setup}
	\caption{Our experiment setup}
	\label{fig:exp_setup}
\end{figure}

\subsection{Parameter Tuning}
Before we begin performing experiments on the full dataset, we first tuned the 
machine learning algorithms using a sample of the forum threads. In the 
following experiments, the threads chosen from our extracted dataset are those 
with a 100 to 1000 posts. This amounted to 97 threads. In each of these 
experiments, we run the algorithm with different parameters, and use the optimal 
one in our final evaluation. 

\subsubsection{Vocabulary size}
Before we begin tuning the other parameters, we start with limiting the feature 
vector when using the content information. This is important as having an 
excessively large feature vector due to a large vocabulary size leads to 
complications due to limited memory size. We restrict our search for a good size 
of vocabulary from $10 \leq K \leq 60$, with increments of 10.  The results are 
shown in Table \ref{table:vocab_exp}. While $K=60$ gives us the best $T$-score, 
we select $K=50$ as this gives us the best $\Prerror$ score.  
\begin{table}
	\footnotesize
\begin{center}
	\begin{tabular}{|l|c|c|c|c|c|c|c|c|}
	\hline
& $T$-score			   &	Visit/Post & 	$\prerror$\\
	\hline
	\input{tables/vocab_exp}
	\hline
	\end{tabular}
\end{center}
	\caption{Experiment results: Varying vocabulary size}
	\label{table:vocab_exp}
\end{table}



\subsubsection{Window size}
Using a combination of feature sets, we experiment with different window sizes, 
$w = 1, 5, 10, 15$.

Performing the experiment using only the $\dt$ values within the window, we 
obtain the results found in Table \ref{tbl:par_tune_dt}. The results show that 
$w=15$ provide the best $T$-score. We must however, keep in mind that its 
Visit/Post ratio is the highest, but also has a higher standard error.

\begin{table}
	\footnotesize
\begin{center}
\begin{tabular}{| l | c | c | c |}
\hline
& $T$-score			   &	Visit/Post & 	$\prerror$\\
\hline
	\input{tables/dt_vec_features}
\hline
\end{tabular}
\end{center}
\caption{Some results}\label{tbl:par_tune_dt}
\end{table}

Using only the content, we perform the same experiment again. Since the size of 
the vocabulary is large, we select the $K = 50$ best tokens to consider using 
Univariate feature selection. This gives us the results in Table 
\ref{tbl:par_tune_content}. The best $T$-score here does not do as well as that 
in the previous experiment. However, it is interesting to note that, again, 
$w=15$ results in the best $T$-score.

\begin{table}
	\footnotesize
\begin{center}
\begin{tabular}{| l | c | c | c |}
\hline
& $T$-score			   &	Visit/Post & 	$\prerror$\\
\hline
	\input{tables/dt_vec_features}
\hline
\end{tabular}
\end{center}
\caption{Some results}\label{tbl:par_tune_content}
\end{table}

For our final experiment for tuning the window size, we combine the various 
feature sets together. We also include the time-context in this experiment, and 
we arrive at the results found in Table \ref{tbl:par_tune_comb}. Again, $w=15$ 
has the best $T$-score, but only with a slight improvement over our first 
experiment.

In any case, this suggests that $w=15$ may be the best window size. In the 
following experiments, this will be our $w$ value.

\begin{table}
	\footnotesize
\begin{center}
\begin{tabular}{| l | c | c | c |}
\hline
& $T$-score			   &	Visit/Post & 	$\prerror$\\
\hline
\input{tables/comb_vec_features}
\hline
\end{tabular}
\end{center}
\caption{Some results}\label{tbl:par_tune_comb}
\end{table}


\subsubsection{Decay factor}
In our discounted sum method, we have to tune the $\alpha$ parameter. We search 
through 0.1 to 0.9 (inclusive) with 0.1 increments to find the best possible 
value for $\alpha$.  We used the combined set of features for this experiment.
The results are shown in Table \ref{tbl:par_tune_decay}.
\begin{table}
	\footnotesize
\begin{center}
\begin{tabular}{| l | c | c | c |}
\hline
& $T$-score			   &	Visit/Post & 	$\prerror$\\
\hline
	\input{tables/alpha_decay}
\hline
\end{tabular}
\end{center}
\caption{Some results}\label{tbl:par_tune_decay}
\end{table}

$\alpha=0.9$ performs the best, but its improvement over the rest of the values 
for $\alpha$ are not by much. Also, note that the $T$-scores do not defer much 
from the previous experiment, although there is a slight improvement.

\subsubsection{Learning rate for Stochastic Gradient Descent}

Because of the scaling factors applied to the sigmoid function, a small change 
in the exponent of $e$ results in huge fluctuations. As such, we need to find a 
small enough learning rate such that the predicted values do not end up at only 
the extremes, but large enough such that the model is adaptive enough to 
``react" to changes.

\begin{table}
	\footnotesize
\begin{center}
\begin{tabular}{| l | c | c | c |}
\hline
& $T$-score			   &	Visit/Post & 	$\prerror$\\
\hline
	\input{tables/learning_rate}
\hline
\end{tabular}
\end{center}
\caption{Some results}\label{tbl:par_tune_learning}
\end{table}

In this experiment, we find that $\eta=5\cdot10^{-8}$ is the best value for the 
learning rate. Also note that this model produces the best results for the 
sample dataset.


So at the end of tuning our feature set and parameters, we have the following 
set of parameters: $K = 50, w = 15, \alpha = 0.9, \eta = 5\cdot10^{-8}$. Using 
these parameters, we run a full evaluation on our dataset.

\section{Experiments}
For our experiments, we selected threads larger than the window size, and have 
enough posts such that we can split each thread into our 3:1 ratio for training 
and testing. As such, the threads are at least 19 posts long, since the training 
set also requires at least two posts in order for us to apply our $\prerror$ 
metric.

This reduces our dataset to a size of 830 threads. We run all three of our 
algorithms on the dataset, and determine if the performance based on the 
$\prerror$ is an improvement over the baseline of revisiting at the average 
rate. The results are seen in Table \ref{tbl:full_eval}.
\begin{table}
	\footnotesize
\begin{center}
\begin{tabular}{| l | c | c | c |}
\hline
& $T$-score			   &	Visit/Post & 	$\prerror$\\
\hline
	\input{tables/full_eval}
\hline
\end{tabular}
\end{center}
\caption{Some results}\label{tbl:full_eval}
\end{table}

It is evident that the scores our methods attain are better than the baseline, 
but only on average. We performed a statistical significance test on our data, 
using the Wilcoxon signed-rank test \cite{wilcoxon1945}. The reason a Student's 
$t$-test could not be employed for this test was due to the non-normal 
distribution of the $T$-score results. The Wilcoxon's signed-rank test, like 
Student's $t$-test, also gives us a $p$-value for statistical significance. For 
each of the three proposed methods, we pair the values with the same thread for 
the baseline, and find the difference in $T$-score, Visit/Post ratio, and 
$\prerror$. We only perform the statistical significance test on $\prerror$.

\begin{table}
	\footnotesize
\begin{center}
\begin{tabular}{| l | c | c | c | l |}
\hline
& $T$-score			   &	Visit/Post & 	$\prerror$ &\\
\hline
	\input{tables/diff_summary}
\hline
\end{tabular}
\end{center}
\caption{Some results}\label{tbl:diff_eval}
\end{table}

We can observe from Table \label{tbl:diff} that first two regression methods 
using SVR perform significantly better than the baseline, while our SGD method 
has a $p$-value of less than 10\%. This begs the question: Do the models perform 
differently for different thread lengths?

\subsubsection{Different thread lengths}
We divide our evaluated threads into 5 different subsets: $0 \leq |P| < 50$,
$50 \leq |P| < 100$, $100 \leq |P| < 150$, $150 \leq |P| < 200$ and $|P| \geq 
200$. We then look at the same evaluation metrics, and see how well these 
methods do on different sizes of training data.

\begin{table}
	\footnotesize
\begin{center}
\begin{tabular}{| l | c | c | c|}
\hline
& $T$-score			   &	Visit/Post & 	$\prerror$ \\
\hline
	\input{tables/summaries/bin_00025_00049}
\hline
	\input{tables/summaries/bin_00050_00099}
\hline
	\input{tables/summaries/bin_00101_00146}
\hline
	\input{tables/summaries/bin_00152_00197}
\hline
	\input{tables/summaries/bin_00202_19378}
\hline
\end{tabular}
\end{center}
\caption{Some results}\label{tbl:bin_eval}
\end{table}
The results, displayed in Table \ref{tbl:bin_eval}, show that the two methods 
using an offline trained model perform poorer on longer thread lengths. This 
suggests that our method that adaptively adjusts its weights based on new 
observations performs better in the long run, with longer threads. Since SGD 
only uses content features, this also suggests that the words found in the 
thread's content can affect the rate of posting on the thread.

The limitations of this, however, are that a sufficient number of observations 
of posts must be made before the model can make good enough predictions.

\section{Recommendations}
Based on our findings, we can make some suggestions as to how an incremental 
crawler could use these techniques to predict when to revisit a site.

The SVR regression method could be used to predict revisitation rates initially, 
since they work better on shorter threads with $|P| < 50$. When the thread grows 
sufficiently, the SGD method could then be used.

This has several benefits. The SVR method requires the entire dataset to be 
present in memory during training, and a new model has to be trained every time 
modifications need to be made.  Using the SGD method, the weight vector could be 
loaded into memory when a new observation is made, the weights updated 
accordingly.
