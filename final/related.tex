\section{Refresh policies for incremental crawlers}
In order to devise such a strategy, we need to predict how often any user may 
update a page. Some work has been done to try to predict how often page content 
is updated, with the aim of scheduling download times in order to keep a local 
database fresh.


We will discuss the \emph{timeliness} of our crawler to maintain the freshness 
of the local database, which refers to how new the extracted information is. Web 
crawlers can be used to crawl sites for user comments and threads for 
postprocessing later. Web crawlers which maintain the freshness of a database of 
crawled content are known as incremental crawlers. Two trade-offs these crawlers 
face cited by \outcite{Yang2009} are \emph{completeness} and \emph{timeliness}.  
\emph{Completeness} refers to the extent which the crawler fetches all the 
pages, without missing any pages. \emph{Timeliness} refers to the efficiency 
with which the crawler discovers and downloads newly created content. We focus 
mainly on timeliness in this project, as we believe that timely updates of 
active threads are more important than complete archival of all threads in the 
forum site.

Many such works have used the Poisson distribution to model page updates.  
\outcite{Coffman1997} analysed the theoretical aspects of doing this, showing 
that if the page change process is governed by a Poisson process 
$\frac{\lambda^k e^{-\lambda \mu}}{k!}$, then accessing the page at intervals 
proportional to $\lambda$ is optimal.

Cho and Garcia-Molina trace the change history of 720,000 web pages collected 
over 4 months, and showed empirically that the Poisson process model closely 
matches the update processes found in web pages \cite{Cho1999}. They then 
proposed different revisiting or refresh policies 
\cite{Cho2003,Garcia-molina2003} that attempt to maintain the freshness of the 
database.

The Poisson distribution were also used in \outcite{Tan2007} and 
\outcite{Wolf2002}. %elaborate!!!!
However, the Poisson distribution is memoryless, and in experimental results due 
to \outcite{Brian2000}, the behaviour of site updates are not. Moreover, these 
studies were not performed specifically on online threads, where the behaviour 
of page updates may be very different from that of static pages.

\outcite{Yang2009}, attempted to resolve this by using the list structure of 
forum sites to infer a sitemap. With this, they reconstruct the full thread, and 
then use a linear-regression model to predict when the next update to the thread 
will arrive. %elaborate!!!

Online forums and bulletins have a logical, hierarchical structure in their 
layout, which typically alerts the user to thread updates by putting threads 
with new replies at the very top of the thread index. Yang's work exploits this 
as well as their linear model to achieve a predicton of when to retrieve the 
pages.
However, this is not so for comments found on blog sites or discussion threads 
in an e-commerce site about a certain product and the lack of these pieces of 
information may result in a poorer estimate, or no estimate at all.


Our perspective is that the available content on the thread at the time of the 
retrieval should also be factored into the model used to predict the page 
updates. Next, we look at some of the related work pertaining to thread content.

\section{Thread content analysis}
While there is little existing work using content to predict page updates, we 
will review some existing work related to analysing thread-based pages which we 
think will aid us in our efforts to do content-based prediction.

\outcite{Wang2011} did work in finding out linkages between forum posts using 
lexical chaining. They proposed a method to link posts using the tokens in the 
posts called $Chainer_{SV}$. While they do analyse the content of the individual 
posts, the paper does not make any prediction with regards to newer posts. The 
methods used to produce a numeric similarities between posts may be used as a 
feature to describe a thread in its current state, but incorporating this into 
our model is non-trivial.

%Kleinberg used Hidden Markov Models to predict ``bursts" in message arrival 
%times \cite{Kleinberg2003}. In his running example, he used email messages, and 
%used time between messages to estimate the states that produced the sequence.  
%While the model may be able to predict what the state is for the next time 
%interval, it does so using the history of message arrival times, and does not 
%take into account the content within the messages themselves.


%One also cannot ignore the fact that social factors play a role when users 
%interact in an online discussion. Granovetter's threshold model for social 
%behaviour may also be useful in describing how the users behave as a whole.

With these related work in mind, we next propose our modelling of a thread as a 
Markov chain, and our approach to solving the problem.


\section{Evaluation metrics}
\outcite{Yang2009} have a metric known as the $T$-score which gives the average 
time a post is made and when the post is received. The lower the $T$-score, the 
better the model. However, the metric does not account for visits which retrieve 
nothing new from the thread.  As such, a crawler that repeatedly crawls the site 
at a frequent rate would do very well.

In \outcite{Georgescul2009}, the authors propose a new scheme for evaluating 
segmentation algorithms, $Pr_{error}$. This metric is the weighted sum of two 
probability counts $Pr_{fa}$ which is the probability that a false alarm 
segmentation is made, and $Pr_{miss}$ which is the probability that a 
segmentation is not made when there should be one. Unfortunately for our 
purposes, the metrics are calculated using the number of ground truths and 
segmentations given a window. As such, it does not account for the ``distance" 
between the ground truths and the segmentation. It also does not allow for the 
predictions to appear after the ground truths, all requirements needed for a 
metric to evaluate timeliness of a model.

\section{Predicting events in social media}
There has been some work done recently in predicting events in social media, and 
in particular, tweets. 
\outcite{Wang2012} dealt with predicting the retweetability of tweets using 
content. They applied two levels of classification, the first level categorising 
tweets into 6 different types: Opinion, Update, Interaction, Fact, Deals and 
Others. This was done using similar techniques as \outcite{Sriram2010} and 
\outcite{Naaman2010}. The Opinion and Update categories are then further 
categorised into another three and two sub-categories each. The authors 
performed this categorisation using labeled Latent Dirichlet Allocation 
%TODO:ITE.

The task in this work, was to predict which of three predefined classes a tweet 
will fall into no retweets, low and high. Our task is slightly more challenging, 
since we are trying to minimise the time from which a post is made to when the 
page is revisited. However, the feature sets used in these works should prove 
useful in our task.  

\section{Forecasting and Machine Learning}
Since the purpose of our envisioned model would be to create a crawler that can estimate the best times to revisit a page, a proper approach to modeling this as a time-series model would be through machine learning techniques.

An interesting approach to forecasting stock prices was presented in \outcite{Cao2003}. The technique involved tweaking conventional SVMs to weigh recent training instances more heavily than older instances. This is a particularly useful idea, since we face the same issue in our task: Recent posts are more descriptive of the current state of the thread, and hence should be more useful in predicting the next post.




