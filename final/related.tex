\section{Refresh policies for incremental crawlers}
In order to devise such a strategy, we need to predict how often any user may 
update a page. Some work has been done to try to predict how often page content 
is updated, with the aim of scheduling download times in order to keep a local 
database fresh.


We will discuss the \emph{timeliness} of our crawler to maintain the freshness 
of the local database, which refers to how new the extracted information is. Web 
crawlers can be used to crawl sites for user comments and threads for 
postprocessing later. Web crawlers which maintain the freshness of a database of 
crawled content are known as incremental crawlers. Two trade-offs these crawlers 
face cited by Yang et. al. 2009 \cite{Yang2009} are \emph{completeness} and 
\emph{timeliness}. \emph{Completeness} refers to the extent which the crawler 
fetches all the pages, without missing any pages. \emph{Timeliness} refers to 
the efficiency with which the crawler discovers and downloads newly created 
content. We focus mainly on timeliness in this project, as we believe that 
timely updates of active threads are more important than complete archival of 
all threads in the forum site.

Many such works have used the Poisson distribution to model page updates.  
Coffman et. al. \cite{Coffman1997} analysed the theoretical aspects of doing 
this, showing that if the page change process is governed by a Poisson process 
$\frac{\lambda^k e^{-\lambda \mu}}{k!}$, then accessing the page at intervals 
proportional to $\lambda$ is optimal.

Cho and Garcia-Molina trace the change history of 720,000 web pages collected 
over 4 months, and showed empirically that the Poisson process model closely 
matches the update processes found in web pages\cite{Cho1999}. They then 
proposed different revisiting or refresh policies 
\cite{Cho2003,Garcia-molina2003} that attempt to maintain the freshness of the 
database.

The Poisson distribution were also used in Tan et. al. \cite{Tan2007} and Wolf 
et. al. \cite{Wolf2002}. %elaborate!!!!
However, the Poisson distribution is memoryless, and in experimental results due 
to Brewington and Cybenko \cite{Brian2000}, the behaviour of site updates are 
not. Moreover, these studies were not performed specifically on online threads, 
where the behaviour of page updates may be very different from that of static 
pages.

Yang et. al. \cite{Yang2009}, attempted to resolve this by using the list 
structure of forum sites to infer a sitemap. With this, they reconstruct the 
full thread, and then use a linear-regression model to predict when the next 
update to the thread will arrive. %elaborate!!!

Online forums and bulletins have a logical, hierarchical structure in their 
layout, which typically alerts the user to thread updates by putting threads 
with new replies at the very top of the thread index. Yang's work exploits this 
as well as their linear model to achieve a predicton of when to retrieve the 
pages.
However, this is not so for comments found on blog sites or discussion threads 
in an e-commerce site about a certain product and the lack of these pieces of 
information may result in a poorer estimate, or no estimate at all.


Our perspective is that the available content on the thread at the time of the 
retrieval should also be factored into the model used to predict the page 
updates. Next, we look at some of the related work pertaining to thread content.

\section{Thread content analysis}
While there is little existing work using content to predict page updates, we 
will review some existing work related to analysing thread-based pages which we 
think will aid us in our efforts to do content-based prediction.

Wang et. al \cite{Wang2011} did work in finding out linkages between forum posts 
using lexical chaining. They proposed a method to link posts using the tokens in 
the posts called $Chainer_{SV}$. While they do analyse the content of the 
individual posts, the paper does not make any prediction with regards to newer 
posts. The methods used to produce a numeric similarities between posts may be 
used as a feature to describe a thread in its current state, but incorporating 
this into our model is non-trivial.

%Kleinberg used Hidden Markov Models to predict ``bursts" in message arrival 
%times \cite{Kleinberg2003}. In his running example, he used email messages, and 
%used time between messages to estimate the states that produced the sequence.  
%While the model may be able to predict what the state is for the next time 
%interval, it does so using the history of message arrival times, and does not 
%take into account the content within the messages themselves.


%One also cannot ignore the fact that social factors play a role when users 
%interact in an online discussion. Granovetter's threshold model for social 
%behaviour may also be useful in describing how the users behave as a whole.

With these related work in mind, we next propose our modelling of a thread as a 
Markov chain, and our approach to solving the problem.


\section{Evaluation metrics}

Big problems... evaluating the performance of a trained model...
Timeliness of model good but does not allow for proper comparisons.. Why?

Other possibilities, Georgescul's method... Pr Error.... 

MAPE..

\section{Predicting events in social media}


\section{Time series prediction}




