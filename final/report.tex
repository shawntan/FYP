\documentclass[hyp,12pt]{socreport}
\usepackage{color, colortbl}
\usepackage{fullpage}
\usepackage{url}
\usepackage{amsmath}
\usepackage{amsfonts}
\usepackage{graphicx}
\usepackage{pgf}
\usepackage{pgfplots}
\usepackage{tikz}
\usepackage{subcaption}
\usepackage[latin1]{inputenc}
\usepackage{verbatim}
\usetikzlibrary{fit}					% fitting shapes to coordinates
\usetikzlibrary{backgrounds}	% drawing the background after the foreground
\usetikzlibrary{arrows,automata,calc,patterns,snakes}

\begin{document}
\newcommand{\putgraphic}[4]{%
\begin{figure}
\begin{center}
	\includegraphics[width=#4\textwidth]{#1}
\end{center}
	\caption{#2} \label{fig:#3}
\end{figure}}


\tikzstyle{background}=[rectangle,
	fill=gray!10,
	inner sep=0.2cm,
	rounded corners=5mm]

\tikzstyle{pvisit}=[circle,
	thick,
	minimum size=0.75cm,
	draw=orange!80,
	fill=orange!25,
	fill opacity=0.5]

\tikzstyle{space}=[circle,
	thick,
	minimum size=0.75cm,
	draw=orange!80,
	draw opacity=0.0,
	fill=blue!0,
	fill opacity=0.0]

\tikzstyle{post}=[circle,
	thick,
	minimum size=0.75cm,
	draw=blue!80,
	fill=blue!20]
% The measurement vector is represented by an orange circle.
\tikzstyle{visit}=[circle,
	thick,
	minimum size=0.75cm,
	draw=orange!80,
	fill=orange!25]



\newcommand{\vocab}{\mathbf{v}}
\newcommand{\dtvec}{\mathbf{t}_\Delta}
\newcommand{\ctxvec}{\mathbf{t}_\text{ctx}}
\newcommand{\dt}{\Delta_t}
\newcommand{\prerror}{Pr_{error}}
\newcommand{\weights}{\mathbf{w}}
\newcommand{\X}{\mathbf{X}}
\newcommand{\post}{\rho}
\renewcommand{\t}{t}
\newcommand{\w}{w}

\newcommand{\Prerror}{Pr_{\text{error}}}

\newcommand{\Z}{\mathbf{Z}}
\newcommand{\W}{\mathbf{W}}

\pagenumbering{roman}
\title{Predicting Web 2.0 Thread Updates}
\author{Shawn Tan}
\projyear{2012}
\projnumber{H079830}
\advisor{A/P Kan Min-Yen}
\deliverables{
	\item Report: 1 Volume
}
\maketitle
\begin{abstract}
With the advent of Web 2.0, sites with forums, or similar thread-based
discussion features are increasingly common. An incremental web crawler aiming 
to maintain a database of up-to-date, extracted information from sites with such 
discussion features must strike a balance between bandwidth usage and freshness 
of data. Our objective: To estimate the arrival times of the next update to such 
threads. We demonstrate three different methods for achieving this using 
regression methods, and make recommendations as to how they can be used in a 
crawling system. We also propose a novel metric for measuring the timeliness of 
such a model that balances between the model's timeliness and bandwidth 
consumption. We show that our methods outperform the baseline, and recommend a 
way to incorporate these methods into an incremental crawler.

\begin{descriptors}
\item World Wide Web
	\begin{itemize}
	\item[] Web searching and information discovery
		\begin{itemize}
		\item[] Web search engines
			\begin{itemize}
			\item[] Web crawling
			\end{itemize}
		\end{itemize}
	\end{itemize}
	\item[] Information Retrieval
		\begin{itemize}
		\item[] Evaluation of retrieval results
			\begin{itemize}
			\item[] Retrieval effectiveness
			\end{itemize}
		\end{itemize}
\end{descriptors}
\begin{keywords}
	web crawlers, revisitation, discussion threads, evaluation metrics
\end{keywords}
\begin{implement}
	python, bash, scikit-learn\nocite{scikit-learn}
\end{implement}
\end{abstract}

\begin{acknowledgement}
I would like to express my gratitude to my parents, for supporting me throughout 
my final year of university. I would also like to thank the following friends: 
Low Wee, Yipeng, Cedric, Davin, Chris and Lan Guan.  Without them, I would not 
have had the moral support during times of stress, nor would I have been able to 
have discussions while fleshing out ideas.

Most of all, I would really like to thank Jesse and A/P Min-Yen Kan, for their 
patient guidance, support, and for the mentorship during the course of this 
project.
\end{acknowledgement}

\listoffigures 
\listoftables
\tableofcontents 

With the advent of Web 2.0, sites with forums, or similar thread-based
discussion features are increasingly common.  
% Jesse: You just made a claim but its not immediately backed-up with any
% supporting statements. Mention of Table 1.1 should probably go here.
Our goal in this thesis
is to create an algorithm that can predict when updates in such
% Jesse: At this juncture, its still unclear what you mean by "updates in such
% threads".
threads will occur.
\begin{table}
	\makebox[\textwidth][c]{
	{\footnotesize
	\begin{tabular}{|l|c|c|c|c|c|c|c|c|c|c|}
		\hline
			\input{tables/web20}
		\hline
	\end{tabular}
	~\\
	}
	}
	{\footnotesize
\caption{Features of popular Web 2.0 sites}
\label{table:web20}
	\begin{tabular}{l l}
		T &= Twitter mentions\\
	 FB L &= Facebook Likes \\
		FB S &= Facebook Shares\\
	G +1 &= Google +1\\
		   L&= Likes (Local) \\
   		DL &= Dislikes (Local) \\
			C &= Comments \\
		PV &= Page Views \\
   Follows &= Site-local feature for keeping track of user's activities
	\end{tabular}
}
\end{table}

Table \ref{table:web20} shows us that many of the popular Web 2.0
sites have comment features. This suggests that content on the web is
increasingly being created by users alongside content providers.

%we are predicting web 2.0 updates
In an increasing number of cases, news travels more quickly through
online community discussions than through traditional media. Users also
typically discuss purchased products bought online on forums,
and companies that want to get timely feedback about their product
should turn to data mined from such sites.

Web crawling is largely IO-bound, a large portion of the time spent crawling is 
% Jesse: awkard phrasing: "a large portion of X is waiting for the server".
% You may want to rephrase "is" to something else.
waiting for the server to supply a response to the request issued by the 
crawler. However, for sites with a large number of pages (like popular forum 
sites), make this infeasible in practice. 
% Jesse: Makes what infeasible? What is the noun you are referring to?
On top of the usual requests it has, 
% Jesse: Another problem with anaphor resolution of "it". You used "sites" not
% "site" earlier.
it then has to deal with repeated requests from such a crawler. Most sites do 
% Jesse: Another anaphor resolution problem for "it". Please check this throughout
% the thesis.
not mind some additional bandwidth, but if it gets excessive, it may be 
% Jesse: here "it" refers to crawling. When was that last mentioned?
construed as a Denial-of-Service attack. At best, the site may deny any further 
% Jesse: "site" or "sites". Use one or the other. Do not mix.
requests from the crawler, and at worst the large number of requests may bring 
% Jesse: I'm not sure if one crawler can bring down an entire site. It takes a
% significant number of requests per second to bring down a site, one that you
% say "is popular", so we can assume that they will have alotted a non-trivial
% amount of bandwidth in the first place. I'm just saying that this assertion
% sounds unrealistic. The crawler doesnt need to bring down the site for this
% situation to be a problem for the site and/or crawler. 
down the site.
% bandwidth use but too much is not a good thing.  See white-hat rules
% about spiders.

A simple method to reduce the amount of polling needed is to use the
% Jesse: "amount of polling needed". I dont believe you have established what
% you mean by "needed". 
average time differences between previous posts to estimate the
% Jesse: "between previous posts" is ambiguous. Between which ones of the
% previous posts?
arrival of the next one, and to abstain from polling until the
estimated time.
% Min: you need to discuss how well this simple baseline does and whether it is 
% actually adopted.
% Jesse: agreed. please remember to include the results for this baseline in
% the experiment section.

A key observation in our work is that the contents of the thread may
also influence the discussion and hence the rate of commenting.  
% Jesse: Can you give a sneak peak (like one enlightening example) for this
% observation?
We
believe that the content of the thread has information that can give a
better estimate of the time interval between the last post and a new
one.

% Min: tie the ``hypothetical'' example with actual posts.  Also, your
% introduction needs to actually carry through to the final
% implementation and experimentation where you use such features to
% estimate properly.
% Jesse: The paragraph below comes too late. The doubt in the reader's mind
% appears immediately after reading "A key observation...", but you only
% mention this example one paragraph later. If you follow Min's suggestion and
% use actual posts, you can refer make label those posts with figure numbers
% and refer to the posts in the text.
For example, a thread in a technical forum about a Linux distribution may start 
out as a question. Subsequent questions that attempt to either clarify or expand 
on the original question may then be posted, resulting in a quick flurry of 
messages. Eventually, a more technically savvy user of the forum may come up 
with a solution, and the thread may eventually slow down after a series of 
messages thanking the problem solver. 

% Min: your introduction needs some work.
% Towards the end you need.  Please restructure the below.
% 1) a clear problem statement
% 2) statements of the contribution of your work
% 3) A navigation paragraph that describes how the rest of the report
% is going to be structured, especially if it deviates from ``normal''
% structure.
Let us define all such thread-based discussion styled sites as forums. Ideally, 
an incremental crawler of such user-generated content should be able to maintain 
a fresh and complete database of content of the forum that it is monitoring.  
However, doing so with the previously mentioned naive method would (1) incur 
excessive costs when downloading un-updated pages, and (2) raise the possibility 
of the web master blocking the requester's IP address.
% Jesse: You need to make a better transition to the next paragraph. How does
% this situation lead you to have the following goal? Why is this goal the
% logical step (or one of the logical steps) to solve the situation. Note that
% Im not asking you to write the answers to these questions as text in the
% report. These questions are meant to make you think about what to actually
% write. Same goes with all of my other questions.

Our high level goal: To come up with a suitable algorithm for revisiting user 
% Jesse: "To come up" is not good English
% Jesse: "for revisiting"? Please use a consistent terminology
discussion threads, based on the discussion content in the thread. In this 
project, we focus on forum threads. We demonstrate three different methods for 
achieving this using regression methods, and also propose a new metric for 
measuring the timeliness of such a model that balances between the model's 
timeliness and bandwidth consumption.
% Jesse: You are burying the lead. It seems to me that you have major
% contributions in this thesis, but you are hiding them inside this tiny
% little paragraph. Make them stand-out. Use a numbered list. Make it clear
% that these are your contributions, especially if they are novel. And if they
% are, you should mention that they're novel whenever appropriate.

In Chapter 2, we explore the related work dealing with predicting web page 
updates and metrics to measure the performance of such algorithms. 
% Jesse: awkward super long noun above "dealing with [noun phrase]"
In Chapter 3, 
we discuss the methods that we have come up with to tackle the problem, while 
% Jesse: "we have come up with" rephrase
Chapter 4 describes the metrics we propose for measuring the performance of 
revisitation algorithms. 
% Jesse: "Revisitation algorithms"? 
In Chapter 5, we perform experiments on a dataset 
extracted from \url{avsforum.com}, 
% Jesse: you could mention a little more about the significance of the results
% and the size of the data set here.
and show that our models perform better than 
an average revisitation baseline. Chapter 6 then discusses our contributions, 
and possible avenues of future work.

% Min: you need a paragraph at the beginning to describe what topics (subsections) you're going to talk wbout and why.
In order to devise such an algorithm, we need to predict how often any user may 
update a page. Some work has been done to try to predict how often page content 
is updated, with the aim of scheduling download times in order to keep a local 
database fresh.

% Min: this chapter needs A LOT OF WORK.  The work you review is piecemeal at best and does not represent a comprehensive survey of the work in the area.  I would cut your losses and try to make a coherent related work chapter instead of including one (mostly irrelevant) work from various tangential areas.

\section{Refresh policies for incremental crawlers}

We first discuss the \emph{timeliness} of our crawler to maintain the freshness 
of the local database, which refers to how new the extracted information is. Web 
crawlers can be used to crawl sites for user comments for 
later post-processing. Web crawlers which maintain the freshness of a database of 
crawled content are known as incremental crawlers. Two trade-offs these crawlers 
face cited by \outcite{Yang2009} are \emph{completeness} and \emph{timeliness}.  
\emph{Completeness} refers to the extent which the crawler fetches all the 
pages, without missing any pages. \emph{Timeliness} refers to the efficiency 
with which the crawler discovers and downloads newly-created content. We focus 
mainly on timeliness in this project, as we believe that timely updates of 
active threads are more important than complete archival of all threads in the 
forum site.

Many such works have used the Poisson distribution to model page updates.  
% Min: why?  Why is it a good fit for the model?
\outcite{Coffman1997} analysed the theoretical aspects of doing this, showing 
that if the page change process is governed by a Poisson process 
$\frac{\lambda^k e^{-\lambda \mu}}{k!}$, then accessing the page at intervals 
proportional to $\lambda$ is optimal.

Cho and Garcia-Molina trace the change history of 720,000 web pages collected 
over four months, and showed empirically that the Poisson process model closely 
matches the update processes found in web pages \cite{Cho1999}. They then 
proposed different revisiting or refresh policies 
\cite{Cho2003,Garcia-molina2003} that attempt to maintain the freshness of the 
database.

The Poisson distribution were also used in \outcite{Tan2007} and 
\outcite{Wolf2002}. %elaborate!!!!
% Min: agreed with the above comment.  Why are these works worth mentioning?  How do they inform your work?
However, the Poisson distribution is memoryless, and in experimental results due 
to \outcite{Brian2000}, the behaviour of site updates are not. Moreover, these 
studies were not performed specifically on online threads, where the behaviour 
of page updates differs from static pages.

\outcite{Yang2009}, attempted to resolve this by using the list structure of 
forum sites to infer a sitemap. With this, they reconstruct the full thread, and 
then use a linear-regression model to predict when the next update to the thread 
will arrive. 
%elaborate!!!
% Min: agreed.  Your description is about the technical, engineering details and not about the qctual work wrt to algorithms

Forums have a logical, hierarchical structure in their 
layout, which typically alerts the user to thread updates by putting threads 
with new replies at the top of the thread index. Yang's work exploits this 
as well as their linear model to achieve a predicton of when to retrieve the 
pages.
However this design pattern is not applied universally; 
% Min: sorry broke your sentence.  Can you fix it?  I've tried.
comments on blog sites 
or e-commerce sites about products do not conform to this pattern.  The lack of such 
information may result in a poorer estimate, or no estimate at all.

The above works all try to estimate the arrival of the next update (comment), but do not leverage an obvious source of information, which is the content of the posts themselves. Our perspective
is that the available thread content can be used to provide a better estimation for predicting page updates. 

Next, we look at some of the related work
pertaining to thread content.

\section{Thread content analysis}
While there is little existing work using content to predict page updates, we 
review existing work related to analysing thread-based pages.  We 
think such work will aid our efforts in content-based update prediction.

% Min: not clear what you mean by "links".  Need to elaborate.  Is this relevant?
\outcite{Wang2011} find links between forum posts using 
lexical chaining. They proposed a method to link posts using the tokens in the 
posts called $Chainer_{SV}$. While they analyse the contents of individual 
posts, the paper does not make any prediction with regards to newer posts. The 
methods used to produce a numeric similarities between posts may be used as a 
feature to describe a thread in its current state, 
% Min: this isn't clear.  You haven't even described what your current model is, so to say it's non-trivial doesn't seem justified.
but incorporating this into 
our model is non-trivial.

%Kleinberg used Hidden Markov Models to predict ``bursts" in message arrival 
%times \cite{Kleinberg2003}. In his running example, he used email messages, and 
%used time between messages to estimate the states that produced the sequence.  
%While the model may be able to predict what the state is for the next time 
%interval, it does so using the history of message arrival times, and does not 
%take into account the content within the messages themselves.


%One also cannot ignore the fact that social factors play a role when users 
%interact in an online discussion. Granovetter's threshold model for social 
%behaviour may also be useful in describing how the users behave as a whole.

% Min: this section is way too short.  You only examined one content related work and not even particularly relevant.  I expect trouble from your examiners.
% Min: This description of your work is out of place.  You can't talk about your work yet, not here.  You could have done an overview of your work in the introduction but it doesn't make sense to have it in the related work section.
With these related work in mind, we next propose our modelling of a thread as a 
Markov chain, and our approach to solving the problem.

\section{Evaluation metrics}
% Min: you need to preface this with a short paragraph to say why evaluation metrics need to be discussed.  I'm not sure this needs to be here; this whole section on related work for evaluation can go with the evaluation section itself.
 
\outcite{Yang2009} proposed a metric for our particular problem of thread update prediction. Known as the $T$-score, it gives the average 
time difference between when a post is made and when the post is retrieved by a crawler. The lower the $T$-score, the 
better the model. However, the metric does not penalize for visits which retrieve 
nothing new from the thread.  As such, a crawler that repeatedly crawls the site 
at a frequent rate would do very well.

Broadening our search for more relevant evaluation metrics that take such wasted bandwidth into account, we turn to related work in the evaluation of segmentation algorithms.  In \outcite{Georgescul2009}, the authors propose a new scheme for evaluating 
segmentation algorithms, $Pr_{error}$. 
% Min: you need to describe how this measure can be applied to your scenario.
This metric is the weighted sum of two 
probability counts $Pr_{fa}$ which is the probability that a false alarm 
segmentation is made, and $Pr_{miss}$ which is the probability that a 
segmentation is not made when there should be one. Unfortunately for our 
purposes, the metrics are calculated using the number of ground truths and 
segmentations given a window. As such, it does not account for the ``distance" 
between the ground truths and the segmentation. It also does not allow for the 
predictions to appear after the ground truths, all requirements needed for a 
metric to evaluate timeliness of a model.

\section{Predicting events in social media}
% Min: Aobo's work doesn't have much to do with your research.  I have no idea why you are including it here.  If relevant at all, it should go into your content analysis section.
There has been some work done recently in predicting events in social media, and 
in particular, tweets.  \outcite{Wang} dealt with predicting the retweetability 
of tweets using content. They applied two levels of classification, the first 
level categorising tweets into 6 different types: Opinion, Update, Interaction, 
Fact, Deals and Others. This was done using similar techniques as 
\outcite{Sriram2010} and \outcite{Naaman2010}. The Opinion and Update categories 
are then further categorised into another three and two sub-categories each. The 
authors performed this categorisation using labeled Latent Dirichlet Allocation 
%TODO:ITE.

The task in this work, was to predict which of three predefined classes a tweet 
will fall into: no retweets, a low number of retweets and a high number of retweets. 
% Min: still unsure of its usefulness.  Strongly suggest moving this to the section on (thread) content analysis.
Our task is slightly more challenging, 
since we are trying to minimise the time from which a post is made to when the 
page is revisited. However, the feature sets used in these works should prove 
useful in our task.  

% Min: I would drop this.  This is like adding 2000+ relevant works and choosing Cao2003 as a representative.  A very bad idea.
\section{Forecasting and Machine Learning}
Since the purpose of our envisioned model would be to create a crawler that can estimate the best times to revisit a page, a proper approach to modeling this as a time-series model would be through machine learning techniques.

An interesting approach to forecasting stock prices was presented in \outcite{Cao2003}. The technique involved tweaking conventional SVMs to weigh recent training instances more heavily than older instances. This is a particularly useful idea, since we face the same issue in our task: Recent posts are more descriptive of the current state of the thread, and hence should be more useful in predicting the next post.

% Min: You need a conclusion to this chapter.
\newcommand{\vocab}{\mathbf{v}}
\newcommand{\dtvec}{\mathbf{t}_\Delta}
\newcommand{\ctxvec}{\mathbf{t}_\text{ctx}}
\newcommand{\dt}{\Delta_t}
\newcommand{\prerror}{Pr_{error}}
\newcommand{\weights}{\mathbf{w}}
\newcommand{\X}{\mathbf{X}}
\newcommand{\post}{\rho}
\renewcommand{\t}{t}
\newcommand{\w}{w}
We aim to predict the amount of time between the arrival of the next post and 
the time the last post in the thread was made. The information available to us 
are the previously made posts that we observe when first visiting the thread.  
The assumption made here is that the thread is not paginated in any way, and a 
single visit to the thread gives us the latest posts without having to traverse 
through the links to the latest page. This is because in practice, we would be 
able to keep track of where the last visited page of the thread was, and reading 
the new posts would incur a few more requests to the thread. This, in comparison 
to constantly hitting the page for updates, would be negligible.

More formally, what we are trying to do is to estimate a function $f$ such that 
given a feature vector $\X$ representative of a window $\post_{t - w + 
1},\post_{t - w + 2},\cdots ,\post_t$, where $\post_t$ represents the $t$-th 
post in the thread, we can approximate $\dt$ with $f(\X)$.  In the following 
sections, we will discuss various methods for estimating $f$.  Various notations 
will be used, a quick reference is provided in Table \ref{table:notations}.


\begin{table}
	\begin{center}
	\begin{tabular}{l l}
	\hline
Notation	&	Description\\
	\hline
$\post$		&	A post\\
$\t$		&	Index of a post in a thread\\
$\w$		&	Number of posts in a window  \\
$\post_\t$	&	The $t$-th post in the thread\\
$\vocab_\t$	&	The frequency count vector of the posts used in the $t$-th 
	post\\
$\dt$		&	Time difference between a post at position $\t$ and a post at 
	position $\t+1$ \\
$\dtvec$	&	Vector of $\dt$s in a given window\\
$\ctxvec$	&	Bit vector representing the day of week, and the hour of day\\
$\X$		&	Feature vector extracted from a window\\
$K$			&	The $K$ best features selected from the vocabulary.\\
	\hline
	\end{tabular}
\end{center}
\caption{Notation reference} \label{table:notations}
\end{table}


In the following sections, we describe the various methods we have to 
approximate $f$.

\section{Baselines}
A simple way of estimating the revisit rate would be to use the average time 
differences given the observed posts, or a training set. In previous work, we 
have seen that if page updates follow a Poisson distribution, then revisiting at 
the Poisson mean would be an optimal revisit policy. %cite paper

In our baseline revisit policy, we took into account the last made post whenever 
we make a visit to the thread, and calculate our next revisit time based on the 
average post intervals added to the timestamp of the previous post. This is in 
contrast to an even simpler revisit policy that just revisits at a constant, 
fixed rate, independent of the posts being made to the thread.

One other way of predicting using average post intervals would be to use the 
concept of a \emph{window}. Averaging out the time differences between the posts 
would intuitively work, because it captures the context of the situation: A 
series of posts with short intervals should mean that the next post would come 
at around the same interval as the few that came before.

In terms of using content for prediction, windows also make sense: Forum users 
view content as paginated posts, so time differences between posts do not affect 
their decision to post. Rather, reading a number of posts together affect 
whether or not the user chooses to reply.

An example of a window ($w=2$) can be seen in Figure \ref{fig:event_series}.

\begin{figure}
	\begin{center}
	
\tikzstyle{background}=[rectangle,
	fill=gray!10,
	inner sep=0.2cm,
	rounded corners=5mm]


\tikzstyle{post}=[circle,
	thick,
	minimum size=1.2cm,
	draw=blue!80,
	fill=blue!20]
% The measurement vector is represented by an orange circle.
\tikzstyle{visit}=[circle,
	thick,
	minimum size=1.2cm,
	draw=orange!80,
	fill=orange!25]

\begin{tikzpicture}[>=latex,text height=1.5ex,text depth=0.25ex]
    % "text height" and "text depth" are required to vertically
    % align the labels with and without indices.
  
  % The various elements are conveniently placed using a matrix:
  \matrix[column sep=0.3cm] {
    % First line: Control input
    	&
		\node (e0)					{};&
		\node (e1)	[post]			{};&
		\node (e2)	[visit]			{};&
		\node (e3)	[post]			{};&
		\node (e4)	[visit]			{};&
		\node (e5)	[post]			{};&
		\node (e6)	[visit]			{};&
		\node (e)					{};&
		&
        \\
	};
    
    % The diagram elements are now connected through arrows:

	\path[-]
		(e0) edge[thick]	(e1)
		\foreach \e in {1,2,3,4,5}{
			let \n1={int(\e+1)} in (e\e) edge[thick] (e\n1)
		}
		(e6) edge[thick]	(e)
	;

	\begin{pgfonlayer}{background}
		\node [background,fit=(e1) (e3)] {};
	\end{pgfonlayer}

\end{tikzpicture}


\begin{tikzpicture}[>=latex,text height=1.5ex,text depth=0.25ex]
    % "text height" and "text depth" are required to vertically
    % align the labels with and without indices.
  
  % The various elements are conveniently placed using a matrix:
  \matrix[column sep=0.3cm] {
    % First line: Control input
    	&
		\node (e0)					{};&
		\node (e1)	[post]			{}; &
		\node (e2)	[visit]			{}; &
		\node (e3)	[post]			{}; &
		\node (e4)	[visit]			{}; &
		\node (e5)	[post]			{}; &
		\node (e6)	[visit]			{}; &
		\node (e)					{};&
		&
        \\
	};
    
    % The diagram elements are now connected through arrows:

	\path[-]
		(e0) edge[thick]	(e1)
		\foreach \e in {1,2,3,4,5}{
			let \n1={int(\e+1)} in (e\e) edge[thick] (e\n1)
		}
		(e6) edge[thick]	(e)
	;

	\begin{pgfonlayer}{background}
		\node [background,fit=(e3) (e5)] {};
	\end{pgfonlayer}

\end{tikzpicture}

	\caption{%
A series of events, posts (blue) and visits (orange).  The diagram demonstrates 
the concept of a window of $w=2$.
}\label{fig:event_series}
	\end{center}
\end{figure}


\section{Performing regression on windows}
% Jesse: I dont understand what role this paragraph plays. It doesnt seem to
% add to explaining your method, which up to this point, still has not been
% explained.
% Shawn: To explain why I didn't do simple linear regression.
Previous work has used linear regression on a number of different features 
extracted from forums \cite{Yang2009}. In their paper, the regressed function 
was used as a scoring function rather than a predictive function. In site of 
this, we attempted to implement the same model, but this resulted in evaluations 
worse than that of the baseline.

We did use some of the features mentioned in the paper: Window posts time 
differences and time context features (bit-vector representations of the day of 
the week and hour of the day). In our own statistics we took from the 
\url{avsforum.com} threads, we have also found that the time of the day the day 
of the week matters when dealing with threads. An example of such a thread can 
be seen in Figure \ref{fig:hr_freq} and Figure \ref{fig:week_freq}, where it can 
be seen that activity on the board is highest at 2 PM, and drops slightly, 
suggesting some type of lunch period, and then goes up again during the early 
evening and at 9 PM, before dropping to its lowest at 3 AM. The weekly graph 
also shows a pattern, showing lower posting frequencies during the weekends, and 
its highest during Thursdays.
\begin{figure}
\begin{center}
\begin{subfigure}[b]{0.45\textwidth}
\includegraphics[width=\textwidth]{diagrams/hoursofday.png}
\caption{The hourly post frequency.}
\label{fig:hr_freq}
	\end{subfigure}
	\begin{subfigure}[b]{0.45\textwidth}
\includegraphics[width=\textwidth]{diagrams/daysofweek.png}
\caption{The daily post frequency. (Monday is 0)}
\label{fig:week_freq}
	\end{subfigure}
\end{center}
\end{figure}


This suggests that the \emph{time context} of which the posts were made are 
important when trying to determine the rate of posts.  As such, we factor in the 
day and hour information into our feature sets as well.

However, this time in stead of linear regression, we used a regression method 
known as Support Vector Regression (SVR), using a radial basis function kernel.  
This method allows the using of different kernels, allowing for better 
estimation of the target function.

The main focus of study in this report was to see if content helps with 
predicting thread updates would produce an improvement. Some of the ways that 
content data were extracted into feature vectors are the following: Word 
frequency, the tf-idf of these words, and Part-of-Speech tags.

We perform the standard preprocessing steps like removing stopwords and tokens 
of length less than three. We also use Porter's stemming algorithm as another 
preprocessing step, before performing a word frequency count. However, the use 
of the full vocabulary of the thread as a feature vector greatly increases the 
time needed to train the model. As such, we used a simple univariate regression 
technique for feature selection, and selected only the $K$ best tokens for 
consideration. Table \ref{table:vocab_exp} shows the results of this experiment.  
	
\begin{table}
	\footnotesize
\begin{center}
	\begin{tabular}{|l|c|c|c|c|c|c|c|c|}
	\hline
	\input{tables/vocab_exp}
	\hline
	\end{tabular}
\end{center}
	\caption{Experiment results: Varying vocabulary size}
	\label{table:vocab_exp}
\end{table}

These feature sets are used in different combinations, with different window 
sizes. The results will be seen in the next chapter.

The methods in this section use features extracted from the current window. A 
model is then trained using these extracted features in order to make a 
prediction. We take a look now at a two other novel methods that we developed.

\section{Discounted sum of previous instances}
 
Posts made further in the history of the thread may have an effect on when the 
latest posts arrive. The magnitude of this effect, however, may diminish over 
time.

Instead of having a finite window for which all posts (in said window) are 
treated equally, why not try to account for all previous posts, but weigh them 
accordingly: the earlier they were made, the less weightage on the prediction 
the post should have.

Following this intuition we used a discounted sum over previous posts' word 
frequency vector:
\[
	\X'_t = \X_t + \alpha \X'_{t-1}
\]
where $\X_t$ is the feature vector at post $t$. $\alpha$ is the \emph{discount 
factor} and satisfies $0 \leq \alpha < 1$.

This new feature vector $\X'_t$ will be used in the same way as before, 
instances of $\X'$ will be regressed with their $\dt$ values. As before, we will 
look at the results for this method in the next chapter.

Up till now, we have looked at methods that treat the model as static -- once 
trained, the model never gets updated during run time. However, this is 
unrealistic due to the fact that over time, different words are popular as a 
direct result of different topics in the real world being popular. In this case, 
these fluctuations may be due to new updates to firmware being released or newer 
models of, say, a stereo set.

\section{Stochastic Gradient Descent}
We also used stochastic gradient descent to estimate the function $f$.  However, 
during runtime, instead of using a static function, we continue to allow $f$ to 
vary whenever new posts and their update times are observed.

Having already attempted using linear regression for this purpose, we have found 
it unsuitable for $f$ to be estimated by a linear function. Such a linear 
function has often resulted in making a negative prediction, and sometimes an 
overly huge one, when given feature vectors that have previously never been 
observed. The function has to be somehow constrained such that the value 
returned never drops below 0, and never predicts something too huge such that 
many posts are missed.



Since $f(\X) > 0$, we used a scaled sigmoid function,
\[
	f(\X) = \frac{\Lambda-\lambda}{1 + e^{\weights \cdot \X}} + \lambda
\]
where $\Lambda$ and $\lambda$ are the scaling factors. This results in $f: 
\mathbb{R}^{|\X|}  \rightarrow (\lambda,\Lambda)$. Bounding the estimation 
function between $\lambda$ and $\Lambda$ allows us to restrict the prediction 
from becoming negative, or, becoming exceedingly huge. For our purposes, we set 
$\lambda = Q_3 + 2.5(Q_{3} - Q_{1})$, where $Q_n$ is the value at the $n$-th 
quartile. A visual interpretation of such a curve can be seen in Figure 
\ref{fig:scaled_sigmoid}.
\begin{figure}
\begin{center}
\begin{tikzpicture}
	\begin{axis}[
		xlabel=$\weights\cdot\X$,
		ylabel=$f(\X)$,
		ytick={0,0.5,1},
		yticklabels={$\lambda$,$\frac{\lambda + \Lambda}{2}$,$\Lambda$}
	]
\addplot {1/(1+ e^-x)}; \end{axis}
\end{tikzpicture}
\end{center}
\caption{Scaled sigmoid curve}\label{fig:scaled_sigmoid}
\end{figure}

The resulting update rule for $\weights$ is then given by,
\[
	\Delta \weights_i = \eta
				\underbrace{\left(\widehat{\dt} - \dt \right)}_{\text{error term}}
				\underbrace{\left( f(\X)(1-f(\X)) \right)}_{\text{gradient}}
						\X_i
\]
which is similar to the delta update rule found in artificial neural networks.  
We omit the scaling factor in the gradient as it is a constant and then 
experiment with various values of $\eta$, the learning rate. 

In this chapter, we have outlined the specific task we will be attempting, to 
try and predict the time from the current last post in the thread to the next.
We have discussed the types of features we will be using, time context features, 
with a focus on content features that consists mainly of the tokens present. We 
have also discussed the concept of a window, and how it could help to make 
predictions better. Also, two novel methods were discussed, and, in the next 
chapter, we will look at how these methods stack up against one another.



\renewcommand{\P}{Pr}
 
\outcite{Yang2009} proposed a metric for our particular problem of thread update prediction. Known as the $T$-score, it gives the average 
time difference between when a post is made and when the post is retrieved by a crawler. The lower the $T$-score, the 
better the model. However, the metric does not penalize for visits which retrieve 
nothing new from the thread.  As such, a crawler that repeatedly crawls the site 
at a frequent rate would do very well.

Broadening our search for more relevant evaluation metrics that take such
wasted bandwidth into account, we turn to related work in the evaluation of
segmentation algorithms.  In \outcite{Georgescul2009}, the authors propose a
new scheme for evaluating segmentation algorithms, $Pr_{error}$. 
% Min: you need to describe how this measure can be applied to your scenario.
This metric is the weighted sum of two 
probability counts $Pr_{fa}$ which is the probability that a false alarm 
segmentation is made, and $Pr_{miss}$ which is the probability that a 
segmentation is not made when there should be one. Unfortunately for our 
purposes, the metrics are calculated using the number of ground truths and 
segmentations given a window. As such, it does not account for the ``distance" 
between the ground truths and the segmentation. It also does not allow for the 
predictions to appear after the ground truths, all requirements needed for a 
metric to evaluate timeliness of a model.


%\begin{figure}
%\begin{center}
%	\includegraphics[width=0.8\textwidth]{diagrams/time_dist.png}
%\end{center}
%\caption{Time distribution for a subset of 97 threads}\label{fig:time_dist}
%\end{figure}
%
\begin{table}
\begin{center}
\begin{tabular}{l l}
	\hline
Notation	&	Description		\\
	\hline
$P$			&	List of posts. \\
$V$			&	List of visits.\\
$T$			&	A thread's $T$-score. \\
	$T_\text{max}$	&	A thread's maximum $T$-score. \\
$t(\post)$	&	Timestamp of the post.\\
	\hline
\end{tabular}
\end{center}
	\caption{Notation used for evaluation metrics}
\end{table}


One of the contributions of this project was also to come up with a good metric 
for measuring the performance of a model that performs predictions. 

% Jesse: the text up to this point can be a subsection in your related work
% chapter. If you choose to move the above into a subsection in the related
% work chapter, it may be helpful in the introduction to mention that part of
% the challenge in this problem is in using an appropraite metric.

Our metric has to be different from traditional methods of measuring 
performance. One example of such a measure is Mean Average Percentage Error 
(MAPE), which we use to measure the performance of the learnt model. This value 
is given by
\[
	\frac{1}{|P|}\sum^{|P|}_{t=1}\left|\frac{f(\X_t) - \dt}{\dt}\right|
\]
where $A_i$ is the actual value, and $F_i$ is the forecasted value for the 
instance $i$. Realistically, the model would not be able to come into contact 
with every possible window, 
% Jesse: "come into contact with every possible window"?     
since chances are it will make an error that causes 
 %explain error in a new section (before this one) ?
it to visit a thread late, causing it to miss two posts or more. This value does 
not reflect how well the model will do in a real-time setting, but gives an idea 
of how far off the model is given a window.

In such measures, the performance of the model is measured on an instance by 
instance basis. 
% Jesse: what is an instance here? Also stock prices are not measured
% day-to-day. So your example makes all this even more confusing.
To give a concrete example, say we are attempting to predict 
stock prices. Given the feature vector as input, we get an estimate of what the 
stock prices will be for, say, the next day. We can then measure the absolute 
difference between what was predicted and the actual amount, and evaluate the 
model based on that.

In our case, we want to know how long it takes before any post made will be 
retrieved by the crawler. We also want to ensure that the model does not choose 
to make too many requests. The rest of the chapter explains in detail how we 
came up with our metric, its advantages and limitations.

% Jesse: Ok, i think you really jumped the gun in this chapter. The
% intro paragraphs needs a lot of work. Imagine you are trying to explain to your
% dad / little brother. Where do you begin? In the first place, this metric is
% supposed to be your contribution in this thesis. So it deserves a coherent
% introduction to its inception and some alotted discussion on its motivation
% and why its different / better than other metrics.

% Jesse: I recommend starting at the beginning: What is the task? Then discuss
% about what are the important aspects to qualify the task? timeliness? Then
% talk about how we would measure these aspects. Then you can talk about the
% metrics. As you can see, by writing the prose this way, there will be a nice
% logical flow from the task to the reasoning and motivation behind what is
% a good / important metric for the task. I dont think you need to rewrite the
% whole chapter, just think about what I said and see how you can tweak or
% re-arrange the introduction. Given the deadline, only do this if you're
% confident on what I mean and the time you will take to do it.

\section{Potential errors}
To be thorough, let us also enumerate the types of errors that a model making 
predictions could encounter.

The model can potentially make a prediction such that the next visit comes 
before the arrival of the next post. The predictions being made are the $\dt$ 
between the posts, rather than the visitation times, hence, it is possible for 
the model to make a prediction that occurs before the current time. An erroneous 
prediction can also cause the crawler to come in before the next post (two, or 
more, visits, but nothing new fetched). Errors of this type waste bandwidth, 
since the crawler will make an unnecessary visit to the page.

Another type of error would have the prediction causing the next visit to come 
some time after a post. Since most predictions are almost never fully accurate, 
there will be some time between the post is made and the page is fetched. These 
errors are still relatively acceptable, but the time difference between the post 
arriving and the visit should be minimised. The visit could also come more than 
one post later. Errors of this kind incur a penalty on the freshness of the 
data, more so than the after one post, especially if the multiple posts are far 
apart time-wise.
%include diagrams

\section{$T$-score, and the Visit/Post ratio}

\begin{figure}
	\begin{center}
	\input{diagrams/t_score_diag}
	\caption{An example of a series of events used in our evaluation.}
	\end{center}
\end{figure}

% Jesse: Did you mention what is blue and what is yellow in the figures?

We also want to know the \emph{timeliness} of the model's visits.  
\outcite{Yang2009} has a metric for measuring this. Taking $\Delta t_i$ as the 
time difference between a post $i$ and it's download time, the timeliness of the 
algorithm is given by
\[
	T = \frac{1}{|P|} \sum^{|P|}_{i=1}\Delta t_i
\]

A good algorithm would give a low $T$-score. However, a crawler that hits the 
site repeatedly performs well according to this metric. The authors account for 
this by setting a bandwidth (fixed number of pages per day) for each iteration 
of their testing. In our experimental results, we also take into account the 
number of page requests made in comparison to the number of posts. %ratio?

%\begin{align*}
%	\begin{array}{l@{\mskip\thickmuskip}l}
%	Pr_{miss} &=  \dfrac{%
%		\sum^{N-k}_{i=1} \left[\Theta_{ref\_hyp} (i,k)\right]%
%	}{%
%		\sum^{N-k}_{i=1} \left[\Delta_{ref} (i,k)\right]%
%	}\\
%	 & \\
%	Pr_{fa} &= \dfrac{%
%		\sum^{N-k}_{i=1} \left[\Psi_{ref\_hyp} (i,k)\right]%
%	}{N-k}
%	\end{array}
%	\begin{array}{l@{\mskip\thickmuskip}l}
%		\Delta_{ref}(i,k) &= \left\{ \begin{array}{l l}
%				1, & \text{if }r(i,k) > 0 \\
%				0, & \text{otherwise} 
%		\end{array} \right.\\
%		\Theta_{ref\_hyp}(i,k) &= \left\{ \begin{array}{l l}
%				1, & \text{if ends with post} \\
%				0, & \text{otherwise} 
%		\end{array} \right.\\
%		\Psi_{ref\_hyp}(i,k) &= \left\{ \begin{array}{l l}
%				1, & \text{if ends with visit}\\
%				0, & \text{otherwise} 
%		\end{array} \right.\\
%	\end{array}
%\end{align*}
%Split the $Pr_{error}$ measure into 3 parts:
%
%Probability of having visits before the first post in the window.
%Probability of having more visits than posts after the first post.
%Probability of having more posts than visits after the first post.

%Viewing the posts made during the thread's lifetime as segmentations of the 
%thread, and the visits made as hypotheses of where the segmentations are, we 
%use the $\prerror$ metric from \outcite{Georgescul2009} as a measure of how 
%close the predictions are to the actual posts. An example can be seen in Figure 
%\ref{prerror}.

%Function that gives me:
%
%Increase in visit to post ratio		increase
%Increase in interval				increase
%Increase between post and visit		increase
%(0,$\infty$ or 1)
%
%Increase between visit and visit	decrease
%($\infty$ or 1,0)
%Increase between post and post		increase
%(0,$\infty$ or 1)
%
%\[
%	\begin{array}{l l}
%	T = \dfrac{%
%		\sum_{e=1}^{|E|-1} \Psi(e_t,e_{t+1}) %
%	}{|E|-1} &
%		\Psi(e_t,e_{t+1}) = \left\{\begin{array}{l l}
%				1-e^{-(e_{t+1} - e_t)}	& \text{if post,visit}\\
%				e^{-(e_{t+1} - e_t)}			& \text{if visit,visit}\\
%		\end{array}\right.
%\end{array}
%\]

\section{Normalising the $T$-score and Visit/Post ratio}
We normalise the $T$-score to get a comparable metric across all the threads. In 
order to do this, we consider again the thread posts and visits as a sequence of 
events. We then define the \emph{lifetime}, denoted as $l$, of the thread as the 
time between the first post and the last post. Any visits that occur after the 
last post are ignored.

We then consider the worst case in terms of timeliness, or misses. This would be 
the case where the visit comes at the end, at the same time as the post. So we 
get a value $T_{\max}$ and $P_{\text{miss}}$ such that
\begin{align*}
	\P_{\text{miss}} &= \frac{T}{T_{\max}}\\
							  &= \dfrac{T}{\left(
					\dfrac{%
			\sum_\post (\max_{\post'}t(\post') - t(\post))}{|P|}\right)} \\
					&= \dfrac{|P| \cdot T}{%
		\sum_\post (\max_{\post'}t(\post') - t(\post))}
\end{align*}


An example can be viewed in Figure \ref{fig:norm_t_score}. Assuming that there 
are no posts before $\post_1$ here, we simply take the usual $T$-score value to 
get $T_\text{max}$
\begin{figure}
\begin{center}
\input{diagrams/norm_t_score_diag}
	\end{center}
\caption{An example of calculating $T_\text{max}$. A visit is assumed at the 
same time as the final post made, and the usual $T$-score metric is 
calculated}\label{fig:norm_t_score}
\end{figure}
It is difficult to consider the worst case in terms of false alarms, or visits 
that retrieve nothing. There could be an infinite number of visits made if we 
are to take the extreme case. In order to get around this, we consider discrete 
time frames in which a visit can occur. Since for this dataset, our time 
granularity is in terms of minutes, we shall use minutes as our discrete time 
frame.
With this simplified version of our series of events, we can then imagine a 
worst-case performing revisit policy that visits at every single time frame.  
Here, we assume all quantities are measured in terms of minutes. This gives us
% Jesse: dangling paragraph for "us"
\[
	\P_{\text{FA}} = \frac{|V|}{(\max_{\post}t(\post)) - |P|}
\]
where $l$ is in units of:our specified discrete time frame. Figure 
\ref{fig:norm_fa_score} shows an example of how $\P_{\text{FA}}$ is calculated.

\begin{figure}
\begin{center}
	\input{diagrams/fa_diag}
\end{center}
\caption{An example of calculating the maximum number of visits given a thread. 
The ratio between the number of visits predicted and the number of visits to the 
thread, and is used as $\P_{\text{FA}}$}
\label{fig:norm_fa_score}
\end{figure}


With these two normalised forms of the original metrics, we can use a weighted 
mean to give a weighted combined form of the two error rates, 
$\P_{\text{error}}$:
\[
	\P_{\text{error}} = \alpha\P_{\text{FA}} + (1-\alpha)\P_{\text{error}}
\]


In the following sections, we will discuss the results of our experiments with 
the various algorithms found in the previous chapter, and measure their 
effectiveness using their $T$-scores and Visit/Post ratio, and comparing them 
using the  $\P_{\text{error}}$ metric.

% Jesse: I think you need a concluding paragraph / section for this chapter.
% Use an unordered / ordered list and say that you have 3 metrics. Summarize
% for each with 1 / 2 sentences on what the metric means (semantically).

In our project, the dataset we used was crawled from 
\url{http://www.avsforum.com/f/}. The forum dealt mainly with Audio-Visual 
equipment, with discussions mainly about technical details, offers and people 
showing off their DIY projects.

The forum was chosen from the list which \outcite{Yang2009} provided in their 
paper. The forum users use mainly proper English, which made removing stopwords 
and stemming simpler.

We crawled 4,158 threads, with a total of 1,002,225 posts. A distribution of how 
the length of threads are distributed can be seen in Figure \ref{fig:len_dist}.  
Threads with 1 to 10 posts already make up half the number of collected threads.
The distribution of the time differences are shown in Figure 
\ref{fig:time_dist}. In both the figures, the right-hand-side cutoff was set at 
1,000 due to the negligible number of items to the right of the cutoff.


\putgraphic{diagrams/len_dist.png}{Distribution of thread 
length}{len_dist}{0.75}
\putgraphic{diagrams/time_dist.png}{Distribution of $\dt$}{time_dist}{0.75}

\section{Experiment setup}
The first 75\% of the thread was used as training data, while the remaining 25\% 
was used as test data. We used Support Vector Regression for this regression 
task, employing a Radial Basis Function kernel as our learning algorithm. 

\begin{figure}
	\input{diagrams/exp_setup}
	\caption{Our experiment setup}\label{fig:exp_setup}
\end{figure}

\subsection{Parameter Tuning}
Before we begin performing experiments on the full dataset, we first tuned the 
machine learning algorithms using a sample of the forum threads. In the 
following experiments, the threads chosen from our extracted dataset are those 
with a 100 to 1000 posts. This amounted to 97 threads.
%TODO: may need to change once results are out.
In each of these experiments, we run the algorithm with different parameters, 
and use the optimal one in our final evaluation. 

\subsubsection{Vocabulary size}
%TODO in progress.
\begin{table}
	\footnotesize
\begin{center}
	\begin{tabular}{|l|c|c|c|c|c|c|c|c|}
	\hline
& $T$-score			   &	Visit/Post & 	$\prerror$\\
	\hline
	\input{tables/vocab_exp}
	\hline
	\end{tabular}
\end{center}
	\caption{Experiment results: Varying vocabulary size}
	\label{table:vocab_exp}
\end{table}



\subsubsection{Window size}
Using a combination of feature sets, we experiment with different window sizes, 
$w = 1, 5, 10, 15$.

Performing the experiment using only the $\dt$ values within the window, we 
obtain the results found in Table \ref{tbl:par_tune_dt}. The results show that 
$w=15$ provide the best $T$-score. We must however, keep in mind that its 
Visit/Post ratio is the highest, but also has a higher standard error.

\begin{table}
\begin{center}
\begin{tabular}{| l | c | c | c |}
\hline
& $T$-score			   &	Visit/Post & 	$\prerror$\\
\hline
	\input{tables/dt_vec_features}
\hline
\end{tabular}
\end{center}
\caption{Some results}\label{tbl:par_tune_dt}
\end{table}

Using only the content, we perform the same experiment again. Since the size of 
the vocabulary is large, we select the $K = 50$ best tokens to consider using 
Univariate feature selection. This gives us the results in Table 
\ref{tbl:par_tune_content}. The best $T$-score here does not do as well as that 
in the previous experiment. However, it is interesting to note that, again, 
$w=15$ results in the best $T$-score.

\begin{table}
\begin{center}
\begin{tabular}{| l | c | c | c |}
\hline
& $T$-score			   &	Visit/Post & 	$\prerror$\\
\hline

	\input{tables/dt_vec_features}
\hline
\end{tabular}
\end{center}
\caption{Some results}\label{tbl:par_tune_content}
\end{table}

For our final experiment for tuning the window size, we combine the various 
feature sets together. We also include the time-context in this experiment, and 
we arrive at the results found in Table \ref{tbl:par_tune_comb}. Again, $w=15$ 
has the best $T$-score, but only with a slight improvement over our first 
experiment.

In any case, this suggests that $w=15$ may be the best window size. In the 
following experiments, this will be our $w$ value.

\begin{table}
\begin{center}
\begin{tabular}{| l | c | c | c |}
\hline
& $T$-score			   &	Visit/Post & 	$\prerror$\\
\hline
\input{tables/comb_vec_features}
\hline
\end{tabular}
\end{center}
\caption{Some results}\label{tbl:par_tune_comb}
\end{table}


\subsubsection{Decay factor}
In our discounted sum method, we have to tune the $\alpha$ parameter. We search 
through 0.1 to 0.9 (inclusive) with 0.1 increments to find the best possible 
value for $\alpha$.  We used the combined set of features for this experiment.
The results are shown in Table \ref{tbl:par_tune_decay}.
\begin{table}
\begin{center}
\begin{tabular}{| l | c | c | c |}
\hline
& $T$-score			   &	Visit/Post & 	$\prerror$\\
\hline
	\input{tables/alpha_decay}
\hline
\end{tabular}
\end{center}
\caption{Some results}\label{tbl:par_tune_decay}
\end{table}

$\alpha=0.9$ performs the best, but its improvement over the rest of the values 
for $\alpha$ are not by much. Also, note that the $T$-scores do not defer much 
from the previous experiment, although there is a slight improvement.

\subsubsection{Learning rate for Stochastic Gradient Descent}

Because of the scaling factors applied to the sigmoid function, a small change 
in the exponent of $e$ results in huge fluctuations. As such, we need to find a 
small enough learning rate such that the predicted values do not end up at only 
the extremes, but large enough such that the model is adaptive enough to 
``react" to changes.

\begin{table}
\begin{center}
\begin{tabular}{| l | c | c | c |}
\hline
& $T$-score			   &	Visit/Post & 	$\prerror$\\
\hline
	\input{tables/learning_rate}
\hline
\end{tabular}
\end{center}
\caption{Some results}\label{tbl:par_tune_learning}
\end{table}

In this experiment, we find that $\eta=5\cdot10^{-8}$ is the best value for the 
learning rate. Also note that this model produces the best results for the 
sample dataset.


So at the end of tuning our feature set and parameters, we have the following 
set of parameters: $K = 50, w = 15, \alpha = 0.9, \eta = 5\cdot10^{-8}$. Using 
these parameters, we run a full evaluation on our dataset.

\section{Experiments}

Results of running Stochastic Gradient Descent
Results of running SVR
Results of running Decay


Analyses of the algo.





With the increasing number of sites leveraging user-generated content, a method 
for predicting the updates of such sites needs to be created in order for an 
incremental web crawler to effectively crawl the site. Our high level goal: to 
predict the posting behaviour of users to such sites.

While this primary goal has many challenges, in this report, we have chosen to 
address challenges specific to forum threads. We want to predict, given content 
of the current thread, the time at which a user would post to the thread. We 
evaluate three different machine learning approaches: Two offline algorithms, 
one that only takes into account only the latest window, and another that 
accounts for past windows, with decreasing weightage. And an online algorithm, 
that uses gradient descent to update its weights every time a new post is 
observed.

Overall, our evaluation shows that our methods work better than the baseline, 
which was to revisit the thread at the average time interval. These are 
promising results, and more can be done to improve upon them. 

There are, however, limitations with the current methods, such as....


\section{Contributions}
We have made the following contributions with our work:
\begin{enumerate}
\item Provide comparable evaluation metrics that can be parameterised, depending 
on the evaluators' priority: freshness or bandwidth.

\item Three different prediction methods using machine learning.
\item The effectiveness of using content and time differences for post 
prediction.
\end{enumerate}

\section{Future Work}

With the proposed methods still having many shortcomings when predicting new 
posts, or providing insight into what causes post arrival times, it leaves much 
room for future work to be done.

\subsection{Using Natural Language Processing (NLP) techniques}
There has been some work done recently in predicting events in social media, and 
in particular, tweets.  \outcite{Wang} dealt with predicting the retweetability 
of tweets using content. They applied two levels of classification, the first 
level categorising tweets into 6 different types: Opinion, Update, Interaction, 
Fact, Deals and Others. This was done using similar techniques as 
\outcite{Sriram2010} and \outcite{Naaman2010}. The Opinion and Update categories 
are then further categorised into another three and two sub-categories each. The 
authors performed this categorisation using labeled Latent Dirichlet Allocation.

The task in this work, was to predict which of three predefined classes a tweet 
will fall into: no retweets, a low number of retweets and a high number of retweets. 

\subsection{Topic modelling}
\cite{Gonzalez2005,Hsu2006}

\subsection{Leveraging context}


\subsection{Using online learning techniques}
Since the purpose of our envisioned model would be to create a crawler that can 
estimate the best times to revisit a page, a proper approach to modeling this as 
a time-series model would be through machine learning techniques.

An interesting approach to forecasting stock prices was presented in 
\outcite{Cao2003}. The technique involved tweaking conventional SVMs to weigh 
recent training instances more heavily than older instances. This is a 
particularly useful idea, since we face the same issue in our task: Recent posts 
are more descriptive of the current state of the thread, and hence should be 
more useful in predicting the next post.


\bibliographystyle{socreport}
\bibliography{report}

\appendix
\chapter{Topic Modelling}\label{top_modeling}
\section{Introduction}
In this project, we plan to use topic modelling to predict the arrival times of 
new posts.

Using Latent Dirichlet Allocation, we want to find a set of topics with 
different time interval distributions. These topics also have a probability from 
transitioning to other topics as the thread continues. What we want to do is, 
based on the current post content, to be able to get a distribution of which 
state/topic the thread is currently in, and using the time interval 
distribution, predict the arrival time of the next post

We shall call the time differences between posts $\dt$. For the rest of this 
report, $\W$ represents a document, in particular, the collection of words that 
the document contains, and $\Z$ represents a topic that a document belongs to.

In the next section, we discuss the details of the project that have been 
completed and some of the complications. The last section will detail the 
remaining tasks left to be done.
\section{Completed Tasks}
\subsection{Data: avsforum.com}
%Window
Our dataset was obtained from from \url{avsforum.com} and stored in a 
tab-delimited format, with each line representing a new post. For each post, we 
extracted the following:
\begin{description}
	\item[Timestamp] Time in seconds-from-epoch format.
	\item[Author] Username of the author of the post.
	\item[Main content] The main textual content of the post.
\end{description}
%TODO REwrite?
For the our purposes, we pre-process the data such that each instance is 
comprised of $k$ posts. We have found that Support Vector Regression gives the 
best results when trying to infer the time of the next post arrival when $k=15$.


\subsection{Latent Dirichlet Allocation}
We used Gibbs sampling to perform LDA for 2 to 10 topics. For a small subset of 
threads in the forum. Treating each window as a document, we attempted to find 
topics that the words in the documents belonged to. The following are some 
interesting results.

Since it is a forum that deals largely with audio visual equipment, a large 
portion of the posts tend to comprise of people seeking to troubleshoot faulty 
equipment.  One of the topics LDA found had the following words:
\begin{verbatim}
player	dvd	get	one	would	use	play	audio	like	samsung	blu	hdmi	
sound	work	problem	ray	also	tri	think	firmwar	set	disc	know	issu	
new	see	updat	look	want	soni	good	better	time	connect	even	
thankstill	need	say	buy	receiv	back	price	output	oppo	make	seem	
video	unit	realli	support	well	soundbar	differcould	vizio	optic	
format	movi	analog	come	cabl	via	got	anyon	much	sinc	denon	
may	sure	display	decod	thing	doesnbest	way	bar	fix	pcm	post	first	
watch	channel	hope	test	digit	said	far	take	panason	right	
anoth	bitstream	return	abl	file
\end{verbatim}

These seem to be the keywords that are consistent with such posts -- when users 
describe `issues' and `problems' with their `firmware', and decide if they 
should `return' the goods.

Another distinct topic that was revealed had to do with positive sentiment, and 
contain words such as `thanks' or `nice'.
\begin{verbatim}
speaker	like	amp	sub	sound	one	####	use	###	would	get	listen	room	
music	good	also	look	power	two	system	realli	set	think	much	
know	post	channel	better	new	pair	audio	great	thank	bass	time	
hear	heard	differ	even	make	see	high	want	say	well	center	
come	level	surround	back	could	still	price	current	wire	need	
home	jamo	dsp	first	play	tri	never	sure	nice	love	end	
output	peopl	design	receiv	thread	focu	run	cabl	front	review	
theater	test	#.#	may	anyon	year	thx	got	hsu	watt	legaci	compar	
subwoof	setup	main	bit	work	right	klipsch	somethdual	impress	guy
\end{verbatim}


However, what we want to study is if these differences in content reflect a 
difference in the arrival times of the next post.

\subsection{Distribution of $\dt$}
Since we have a way of obtaining $\P{\Z}{\W}$, we need $\P{\dt}{\Z}$ in order to 
have a way to predict $\dt$. To have a sense of what this looks like, we have 
binned the $\dt$ into bins of 20 minutes, and plotted their frequency based on 
the topics we extracted. A document is said to be coming from topic $z$ if 
$\max_{z'} \P{z'}{\W} = z$
The idea was to see if there was a distinguishable $\dt$ distribution given 
different topics. See Figure \ref{fig:time_dist}.
\begin{figure}
\centering
\begin{subfigure}[b]{0.5\textwidth}
	\centering
	\includegraphics[width=\textwidth]{/home/shawn/Desktop/predict-forum-pgm/graphs/hist_4_topics/001.png}
	\caption{Topic 1}
\end{subfigure}%
~ %add desired spacing between images, e. g. ~, \quad, \qquad etc. 
  %(or a blank line to force the subfigure onto a new line)
\begin{subfigure}[b]{0.5\textwidth}
	\centering
	\includegraphics[width=\textwidth]{/home/shawn/Desktop/predict-forum-pgm/graphs/hist_4_topics/002.png}
	\caption{Topic 2}
\end{subfigure} \\
\begin{subfigure}[b]{0.5\textwidth}
	\centering
	\includegraphics[width=\textwidth]{/home/shawn/Desktop/predict-forum-pgm/graphs/hist_4_topics/003.png}
	\caption{Topic 3}
\end{subfigure}%
~%
 %
\begin{subfigure}[b]{0.5\textwidth}
	\centering
	\includegraphics[width=\textwidth]{/home/shawn/Desktop/predict-forum-pgm/graphs/hist_4_topics/004.png}
	\caption{Topic 4}
\end{subfigure}
\caption{Time distributions for $k = 4$}\label{fig:time_dist}
\end{figure}

Unfortunately, the time distributions did not appear very distinct from each 
other. More tests need to be performed to see if using the expected value from 
this currently gathered data will reflect anything.
\section{Work In Progress}
Including the previous post topic, we can make a better estimate of the $\dt$ 
distribution of the current post.
\[
	p(\Z_t|\Z_{t-1},\W)=
	\frac{p(\Z_{t-1}|\Z_t) \cdot p(\Z_t|\W_t) \cdot p(\W_t)}{
	p(\Z_{t-1})\cdot p(\W)}
\]

This essentially makes it similar to a Hidden Markov Chain, since the document 
topic is dependent on the previous document's topic, and the observations 
produced are the text. This is, however, dependent on whether the distribution 
of $\dt$ is different for each topic.

As seen in Figure \ref{fig:time_dist}, which shows one example of LDA produced 
topics, the time distributions are not distinguishable. This may suggest that 
this approach may not be feasible, and more work needs to be done.


\end{document}
