\tikzstyle{state}=[circle,
	thick,
	minimum size=1.2cm,
	draw=blue!80,
	fill=blue!20]
% The measurement vector is represented by an orange circle.
\tikzstyle{measurement}=[circle,
	thick,
	minimum size=1.2cm,
	draw=orange!80,
	fill=orange!25]

\tikzstyle{background}=[rectangle,
	fill=gray!10,
	inner sep=0.2cm,
	rounded corners=5mm]

\begin{tikzpicture}[>=latex,text height=1.5ex,text depth=0.25ex]
    % "text height" and "text depth" are required to vertically
    % align the labels with and without indices.
  
  % The various elements are conveniently placed using a matrix:
  \matrix[row sep=2cm,column sep=0.3cm] {
    % First line: Control input
    	&
		\node (prev_state)					{$\cdots$};&
		&
        &
        \node (curr_state)	[state]			{$q_t$}; &
        &
		&
		\node (next_state)					{$\cdots$};&
		&
        \\
		&
		&
		\node (obs_text)	[measurement]	{$\Delta_t$};&
		&
		&
		&
		\node (obs_time)	[measurement]	{$\mathbf{v}$};&
		&
        \\
	};
    
    % The diagram elements are now connected through arrows:

	\path[->]
		(prev_state) edge[thick]	(curr_state)
		(curr_state) edge[thick]	(next_state)
		(curr_state) edge			(obs_text)
		(curr_state) edge			(obs_time)
	;
 	  \begin{pgfonlayer}{background}
        \node [background,
					fit=\onslide<1>{(prev_state) (next_state)},
                    label=left:Hidden states:] {};
    \end{pgfonlayer}
\end{tikzpicture}
